\documentclass{article}
\title{Automata}
\author{Jeroen F. J. Laros}
\date{\today}
\usepackage{/home/jeroen/studie/dw/graphs/graphs}
\usepackage{amsfonts, amssymb, amsthm}
\frenchspacing
\setlength{\parindent}{0pt}
\begin{document}

\renewcommand{\qedsymbol}{$\blacksquare$}
\newcommand{\bs}{\begin{small}}
\newcommand{\es}{\end{small}}
\newcommand{\bS}{\begin{tiny}}
\newcommand{\eS}{\end{tiny}}
\newcommand{\monoit}[1]{\texttt{\textit{#1}}}

\newtheorem{theorem}{Theorem}[subsection]
\newtheorem{lemma}[theorem]{Lemma}
\newtheorem{corollary}[theorem]{Corollary}

\theoremstyle{definition}
\newtheorem{example}[theorem]{Example}
\newtheorem{definition}[theorem]{Definition}
\newtheorem{remark}[theorem]{Remark}

\maketitle

\section{Introduction}
Automatic sequences are connected in a fundamental way with substitutions of
\emph{constant length}.

\begin{definition}[Finite automaton] \label{def:finite_automaton}
A finite automaton is a 5-tuple $A = \{\mathcal{S}, \Delta, \delta, I, F\}$
in which:
\begin{itemize}
\item $\mathcal{S}$ is the finite set of states.
\item $\Delta$ is the finite alphabet of labels.
\item $\delta \subseteq \mathcal{S} \times \Delta \times \mathcal{S}$ is the 
      collection of transitions.
\item $I \subseteq \mathcal{S}$ is the collection of initial states.
\item $F \subseteq \mathcal{S}$ is the collection of final states.
\end{itemize}
\end{definition}

A finite automaton is represented as a directed graph with a set of vertices
$\mathcal{S}$ called \emph{states}, a set of edges $\delta$ called 
\emph{transitions}
and specially marked subsets of states $I$ and $F$, being the initial and final
states.

For our purposes, we will restrict ourselves to automata that have
$F = \mathcal{S}$ as the set of final states, so we will leave notion $F$
out unless stated otherwise. The states in $I$ are marked with a $\Uparrow$.

\begin{definition}[Finiteness] \label{def:finiteness}
An automaton is called \emph{finite} if $\mathcal{S}$ is finite.
\end{definition}

\begin{definition}[Determinism] \label{def:determinism}
An automaton is called \emph{deterministic} if the following two conditions 
hold:
\begin{itemize}
\item $\exists (p \in I) \forall (q  \in \mathcal{S})
\{q = p \lor q \notin I\}$
\item $\forall (p \in \mathcal{S}) \forall (a \in \Delta)
\forall (q \in \mathcal{S}) \forall (r \in \mathcal{S})
\{(p, a, q) \in \delta \land (p, a, r) \in \delta \Rightarrow q = r\}$.
\end{itemize}
\end{definition}

In other words:
\begin{itemize}
\item There must be one and only one initial state.
\item There can be no two branches with the same label coming from the same 
  state.
\end{itemize}
% Picture of non-deterministic automaton. {{{1
\begin{graph}(0, 3)(-4, -1.5)
  \graphnodecolour{1}
  \graphnodesize{1}
  \roundnode{s1}(-2, 0) \nodetext{s1}(0, 0){$a$}
  \roundnode{s2}(0, 0) \nodetext{s2}(0, 0){$b$}

  \dirloopedge{s1}{50}(-1, 0) \freetext(-3.6, 0){0}
  \diredge{s1}{s2} \edgetext{s1}{s2}{0}

  \freetext(-2, -0.7){$\Uparrow$}
  \freetext(-2.5, -1.2){non-deterministic}
\end{graph}
%}}}1
% Picture of deterministic automaton. {{{1
\begin{graph}(0, 3)(-9, -1.5)
  \graphnodecolour{1}
  \graphnodesize{1}
  \roundnode{s1}(-2, 0) \nodetext{s1}(0, 0){$a$}
  \roundnode{s2}(0, 0) \nodetext{s2}(0, 0){$b$}

  \dirloopedge{s1}{50}(-1, 0) \freetext(-3.6, 0){0}
  \diredge{s1}{s2} \edgetext{s1}{s2}{1}

  \freetext(-2, -0.7){$\Uparrow$}
  \freetext(-2.5, -1.2){deterministic}
\end{graph}
%}}}1

\begin{definition}[Completeness] \label{def:completeness}
An automaton is \emph{complete} (or total) if the following condition holds:
\begin{itemize}
\item $\forall (p \in \mathcal{S}) \forall (a \in \Delta)
\exists (q \in \mathcal{S}) \{(p, a, q) \in \delta\}$
\end{itemize}
\end{definition}

In other words: Each state must have $|\Delta|$ outgoing branches.

% Picture of non-complete automaton. {{{1
\begin{graph}(0, 3)(-4, -1.5)
  \graphnodecolour{1}
  \graphnodesize{1}
  \roundnode{s1}(-2, 0) \nodetext{s1}(0, 0){$a$}
  \roundnode{s2}(0, 0) \nodetext{s2}(0, 0){$b$}

  \dirloopedge{s1}{50}(-1, 0) \freetext(-3.6, 0){0}
  \diredge{s1}{s2} \edgetext{s1}{s2}{1}

  \freetext(-2, -0.7){$\Uparrow$}
  \freetext(-2.5, -1.2){non-complete}
\end{graph}
%}}}1
% Picture of complete automaton. {{{1
\begin{graph}(0, 3)(-9, -1.5)
  \graphnodecolour{1}
  \graphnodesize{1}
  \roundnode{s1}(-2, 0) \nodetext{s1}(0, 0){$a$}
  \roundnode{s2}(0, 0) \nodetext{s2}(0, 0){$b$}

  \dirloopedge{s1}{50}(-1, 0) \freetext(-3.6, 0){0}
  \dirloopedge{s2}{50}(1, 0) \freetext(1.6, 0){0,1}
  \diredge{s1}{s2} \edgetext{s1}{s2}{1}

  \freetext(-2, -0.7){$\Uparrow$}
  \freetext(-2.5, -1.2){complete}
\end{graph}
%}}}1

\section{$k$-automata}
The $k$-automaton (not a 2-tape automaton or transducer, as stated in
\cite{Fogg} page 12, but rather a Moore automaton) is a finite, 
deterministic and complete automaton with $|\Delta| = k$, expanded with an 
output function or exit map.

So a $k$-automaton is an automaton with states $\mathcal{S}$. Each state has
$k$ outgoing branches, labeled $0, \ldots, k - 1$. Furthermore, there is only
one initial state. Each state also has an output function. For $k$-automata
we use the following notation.

\begin{definition}[$k$-automaton] \label{def:k-automaton}
For $k \in \mathbb{N} \ge 2$ denote by
\begin{itemize}
\item $\mathcal{S}$: A finite set of states. There is a unique 
      $\iota \in \mathcal{S}$ called the initial state.
\item $\Delta$: $k$ labels indicated by integers from 0 to $k - 1$.
\item $\mathcal{L}$: The input language, in this case always $\Delta^*$.
\item $\sigma: S \to S^k$: The substitution such that 
      $\sigma(a) = \sigma_0(a) \sigma_1(a) \ldots \sigma_{k - 1}(a)$ indicates 
      the endpoints of the transitions starting from $a$ with labels 
      $0, 1, \ldots, k - 1$, respectively.
\item $Y$: The output alphabet. We usually take $Y = \mathcal{S}$.
\item $\varphi$: A function from $\mathcal{S}$ to $Y$ called the exit map. We
      usually take $\varphi =$ Id.
\end{itemize}
\end{definition}

\begin{remark} \label{rem:output}
Normally, an automaton accepts a string if it is in
a final state after having read the complete input. In this case, the final 
state results (via $\varphi$) in an output.
\end{remark}

\begin{remark} \label{rem:generate_accept}
Note that we use automata to generate a sequence, not to recognize one as is
usual in computer science. Further on we shall prove that most of the generated
sequences can not be recognized by finite automata.
\end{remark}

\begin{example} \label{ex:ab_star}
\begin{eqnarray*}
&&k = 2\\
&&\mathcal{S} = \{a, b\}, \iota \in \mathcal{S} = \{a\}\\
&&\Delta = \{0, 1\}\\
&&\sigma:\{a, b\} \to \{a, b\}^*, \sigma(a) \to ab,
\sigma(b) \to ab\\
&&Y = \{a, b\}\\
&&\varphi: \varphi(a) \to a, \varphi(b) \to b
\end{eqnarray*}
The automaton is given by the directed graph

% Picture of mod-2 automaton. {{{1
\begin{graph}(0, 3)(-4, -1.5)
  \graphnodecolour{1}
  \graphnodesize{1}
  \roundnode{s1}(-2, 0) \nodetext{s1}(0, 0){$a$}
  \roundnode{s2}(0, 0)  \nodetext{s2}(0, 0){$b$}

  \dirloopedge{s1}{50}(-1, 0) \freetext(-3.6, 0){0}
  \dirbow{s1}{s2}{.2} \bowtext{s1}{s2}{.2}{1}
  \dirbow{s2}{s1}{.2} \bowtext{s2}{s1}{.2}{0}
  \dirloopedge{s2}{50}(1, 0) \freetext(1.6, 0){1}

  \freetext(-2, -0.7){$\Uparrow$}
\end{graph}\\
%}}}1
The substitution $\sigma$ (of constant length 2) has only one fixed point;\\
\\
\monoit{u = ababababababababababababababababababababababababab\ldots}
$= (ab)^\mathbb{N}$.\\
\\
If we take the base 2 expansion of an integer (let us say the decimal
number 22) and feed it to the automaton above, the automaton receives the
digits 10110 and will be in state $a$ when it starts.\\
After reading the first digit, it will be in state $b$, after reading the
second one the automaton will be in state $a$. The table below shows the path
taken:\\
\\
\begin{tabular}{c|c|c|l}
state & transition & next state & tail\\
\hline
$i=a$ & 1 & $b$ & 0110\\
$b$   & 0 & $a$ & 110\\
$a$   & 1 & $b$ & 10\\
$b$   & 1 & $b$ & 0\\
$b$   & 0 & $a$ &\\
$a$   &   &   &
\end{tabular}\\
\\
It will be clear that this automaton will always reach state $a$ if the last 
digit is 0 and state $b$ if it is 1, so it maps any integer $n$ to $a$ if and 
only if $n \mathrm{\ mod\ } 2 = 0$ and to $b$ if and only if 
$n \mathrm{\ mod\ } 2 = 1$.

Thus if we feed the binary sequence 0, 1, 10, 11, 100, \ldots of non-negative 
integers to the automaton, we will get the sequence $u$.
\end{example}

Example \ref{ex:ab_star} shows a duality between automata and words invariant 
under substitutions. We shall study this duality in the sequel.

\subsection{Direct reading}
\begin{definition}[Letter-to-letter projection] \label{def:letter-to-letter}
Consider a map from a finite alphabet $\mathcal{A}$ to an other finite alphabet
$\mathcal{B}$. This map extends in a natural way (by concatenation) to a map 
from $\mathcal{A}^* \cup \mathcal{A}^\mathbb{N}$ to
$\mathcal{B}^* \cup \mathcal{B}^\mathbb{N}$.
\end{definition}

\begin{definition}[Direct reading] \label{def:direct_reading}
A sequence $u = (u_n)_{n \in \mathbb{N}}$ with values in $Y$ is $k$-automatic 
in direct reading if it can be generated by a $k$-automaton as follows: For 
$n = 0, 1, \ldots$
\begin{itemize}
\item let $\sum_{i=0}^j n_i k^i (n_j \neq 0)$ be the base $k$ expansion of an
      integer $n$.
\item initialize the automaton and feed it with the sequence 
      $n_j, \ldots, n_ 1, n_0$, note that the most significant digit is read 
      first.
\item put $u_n = \varphi(a(n))$, if the automaton is in state $a(n)$ after all
      letters have been read.
\end{itemize}
\end{definition}

\begin{theorem} \label{thm:direct_reading}
A sequence $u$ is $k$-automatic in direct reading if and only if $u$ is the 
image of a letter-to-letter projection of a fixed point of a substitution of 
constant length $k$.
\end{theorem} 

\begin{proof}
$\Leftarrow$ Let $u$ be a fixed point of a substitution $\sigma$ of length
$k$ over the alphabet $\mathcal{A}$. We construct a $k$-automaton in direct 
reading. Let $\mathcal{S = A}$. We make a transition from $a$ to $b$ labeled 
$i$ if $b$ occurs in $\sigma(a)$ at position $i + 1$. Take 
$\iota \in \mathcal{S} = u_0$ as the initial state. Take $\varphi =$ Id.
Let $\sum_{i = 0}^t n_i k^i$ be the $k$-adic expansion of $n$. We start from 
$u(0)$, then go to the $(n_t + 1)$--th letter of $\sigma(u(0))$, denoted by 
$a_1$, then go to the $(n_{t - 1} + 1)$--th letter of $\sigma(a_1)$, which is 
also the $(kn_t + n_{t - 1} + 1)$--th letter of $\sigma^2(u(0))$, and so on.
After $t$ steps we arrive at the $(n + 1)$--th letter of 
$\sigma^{t + 1}(u_0)$, which is $u(n)$. We have constructed an automaton that 
generates the sequence $u$ in direct reading. If $v$ is the image of a 
letter-to-letter projection $\varphi: \mathcal{A} \to \mathcal{B}$ of the 
fixed point $u = (u(n))_{n \in \mathbb{N}} \in \mathcal{A}^\mathbb{N}$ of a 
substitution $\sigma$ of constant length $k$ defined on the alphabet 
$\mathcal{A}$. Then $v$ is generated by the same automaton that generated $u$, 
but with the projection $\varphi$ as output function.

$\Rightarrow$ Let $u$ be a sequence generated by a $k$-automaton in direct
reading. Let $\mathcal{S}$ be the set of states of the automaton and let
$f_0, \ldots, f_{k - 1}$ be the transition maps (hence 
$f_i: \mathcal{S \to S}$ maps state $j$ to the state which is reached by 
following transition $i$ from state $j$ for every $j \in \mathcal{S}$). Define 
the substitution of constant length $\sigma = f_0 \dots f_{k - 1}$ over 
$\mathcal{S}$. Let $v$ be the fixed point of $\sigma$ beginning with the 
initial state $\iota$. It is easily checked that the sequence $u$ is the image 
by the output function $\varphi$ of the fixed point $v$. 
\end{proof}

\begin{remark}
When using direct reading we map the initial state $\iota$ onto itself with 
label 0 by default. This is a direct consequence of the existence of the fixed
point of a substitution.
\end{remark}

\begin{example}[The Cantor sequence]
The Cantor sequence is defined as the invariant word under the substitution: 
$\sigma: \sigma(a) \to aba, \sigma(b) \to bbb$. With $a$ as the
initial letter, it gives the following fixed point.\\
\\
\monoit{u = ababbbababbbbbbbbbababbbababbbbbbbbbbbbbbbbbbbbbbbbbb\ldots}\\
\\
The following automaton is associated with this
substitution.

% Picture of cantor automaton. {{{1
\begin{graph}(0, 3)(-4, -1.5)
  \graphnodecolour{1}
  \graphnodesize{1}
  \roundnode{s1}(-2, 0) \nodetext{s1}(0, 0){$a$}
  \roundnode{s2}(0, 0)  \nodetext{s2}(0, 0){$b$}

  \dirloopedge{s1}{50}(-1, 0) \freetext(-3.6, 0){0,2}
  \diredge{s1}{s2} \edgetext{s1}{s2}{1}
  \dirloopedge{s2}{50}(1, 0) \freetext(1.6, 0){0,1,2}

  \freetext(-2, -0.7){$\Uparrow$}
\end{graph}
%}}}1

Let us denote the set of integers $n$ such that the $(n + 1)$--th letter of the
Cantor sequence is $a$ by $\mathbb{C}_a$.The automaton above suggests that 
$\mathbb{C}_a$ is given by
\begin{displaymath}
  \mathbb{C}_a = \Big\{n \in \mathbb{N}; n = \sum_{i \ge 0} n_i3^i,
  \mathrm{\ with\ } \forall(i \ge 0): n_i \in \{0, 2\}\Big\}.
\end{displaymath}
The following argument shows that this is true. Let 
$n = \sum_{i = 0}^t n_i 3^i$ with $n_i \in \{0, 1, 2\}$ for all $i$. Consider 
the mapping $\tau: \{0, 1, \ldots, t\} \to \{a, b\}$ with 
$\tau(j) = u(\sum_{i = 0}^j n_i 3^i)$. By definition of $\sigma$ we have 
$\tau(j + 1) = a$ if and only if $\tau(j) = a$ and $n_{j + 1} \in \{0, 2\}$.

Compare the Cantor fractal with the Cantor word:

% Picture of the Cantor set. {{{1
\ \\
\verb#   0      0.1       0.2      1  |   0     0.1         0.2     1#\\
\verb#---------------------------------------------------------------#\\
\verb#   ===========================  |               #\monoit{a}\\
\verb#   =========         =========  |      #\monoit{a} { } { } { } { } { } { }\monoit{b} { } { } { } { } { } { }\monoit{a}\\
\verb#   ===   ===         ===   ===  |   #\monoit{a { }b { }a { }b { }b { }b { }a { }b { }a}\\
\verb#   = =   = =         = =   = =  |  #\monoit{ababbbababbbbbbbbbababbbaba}
%}}}1
\end{example}

\subsection{Reverse reading}
\begin{definition}[$k$-kernel] \label{def:k-kernel}
Let $N_k(u)$ be the set of subsequences of the sequence 
$(u(n))_{n \in \mathbb{N}}$ defined by:
\begin{displaymath}
N_k(u) = \{(u(k^ln + r))_{n \in \mathbb{N}}; l \ge 0; 0 \le r \le k^l - 1\}.
\end{displaymath}
\end{definition}

\begin{example} \label{ex:k-kernel}
The kernel of the Cantor sequence is given by\\
\\
\monoit{ababbbababbbbbbbbbababbbababbbbbbbbbbbbbb\ldots}
$(l = 0) \Rightarrow u(3^0 n + 0)_{n \in \mathbb{N}}$,\\
\monoit{bbbbbbbbbbbbbbbbbbbbbbbbbbbbbbbbbbbbbbbbb\ldots}
$(l = 1) \Rightarrow u(3^1 n + 1)_{n \in \mathbb{N}}$.\\
\end{example}

\begin{definition}[Reverse reading] \label{def:reverse_reading}
This definition is analogue to Definition \ref{def:direct_reading}, but it
feeds the sequence $n_0, n_1, \ldots, n_j$ to the automaton.
\end{definition}

\begin{theorem} \label{thm:reverse_reading}
A sequence $u \in \mathcal{A}^\mathbb{N}$ is $k$-automatic in reverse reading 
if and only if the $k$-kernel $N_k(u)$ of the sequence $u$ is finite.
\end{theorem}

\begin{proof}
$\Leftarrow$ Suppose the $k$-kernel of a sequence $u$ is finite. Let
$\overline{a}_1, \ldots, \overline{a}_d$ be the sequences of $N_k(u)$. Take
$\mathcal{S} = \{a_1, \ldots, a_d\}$ as the finite set of $d$ states in 
bijection with $N_k(u)$ with $a_1$ as initial state. For any integer 
$r \in \{0, \ldots, k - 1\}$, define the map $r : \mathcal{S} \to \mathcal{S}$ 
by associating the state corresponding with  
$(\overline{a}_i(kn + r))_{n \in \mathbb{N}}$ to $a_i$. Let 
$n = \sum_{i = 0}^j n_i k^i$ be the base $k$ expansion of an integer $n$, with 
$n_j \ne 0$. We define the map $n$ from
$\mathcal{S}$ to $\mathcal{S}$ by $n(a_i) = n_j(n_{j - 1}( \ldots (n_0(a_i))))$
if $n \ne 0$, otherwise the map 0 is the identity. It follows by induction that
$r(a_i)$ is in bijection with $(u(k^{j + 1} n + r))_{n \in \mathbb{N}}$. 
Hence if $r(a_1) = s(a_1)$ then $u_r = u_s$, the first terms of the
two corresponding subsequences. Now we define the output 
function $\varphi$ by $\varphi(a_i) = u_r$ if 
$r(a_1) = a_i$. Therefore the sequence $u$ is generated by the automaton 
in reverse reading.

$\Rightarrow$ Suppose $A$ is a finite $k$-automaton with initial state $\iota$
which generates $u$. The subsequence $(u(k^ln + r))_{n \ge 0}$, where
$l \ge 0$ and $0 \le r < k^l$, is generated by $A$ with the initial state
$\overline{r}(\iota)$, where $\overline{r}$ is the word of $l$ letters
obtained by concatenating in front of the base $k$ expansion of $r$ as many
zeros as necessary. Because $A$ has a finite number of states, there is only a
finite set of subsequences. 
\end{proof}

The proof of Theorem \ref{thm:reverse_reading} gives a method for constructing 
a $k$-automaton in reverse reading corresponding to a given sequence $u$ if $u$ 
is $k$-automatic in reverse reading. Take the $k$-kernel of $u$ and let 
$\mathcal{S}$ be in bijection with the sequences in the kernel. Now add 
transitions from state $a$ to state $b$ labeled $r$ 
($r \in \{0, 1, \ldots, k - 1\}$) if the sequence associated with $b$ is the 
subsequence $(\overline{a}(kn + r))_{n \in \mathbb{N}}$. Make $a_0$ the initial
state and make all states that have incoming transitions (starting from that 
initial state) final states.

\begin{example} \label{ex:reverse_reading}
The fixed point of the substitution $a \to ab$, $b \to ab$,
which is $abababab\ldots$ has as 2-kernel\\
\\
\monoit{ababababababababababababababababababa\ldots}
$= (ab)^\mathbb{N} = (u(2^0 n + 0))_{n \in \mathbb{N}}$,\\
\monoit{aaaaaaaaaaaaaaaaaaaaaaaaaaaaaaaaaaaaa\ldots}
$= a^\mathbb{N} = (u(2^1 n + 0))_{n \in \mathbb{N}}$,\\
\monoit{bbbbbbbbbbbbbbbbbbbbbbbbbbbbbbbbbbbbb\ldots}
$= b^\mathbb{N} = (u(2^1 n + 1))_{n \in \mathbb{N}}$.\\
\end{example}

By taking $l = 0$, we find $(2^0 n + 0)_{n \in \mathbb{N}}$. For $l = 1$ we
obtain two sequences: $(2^1 n + 0)_{n \in \mathbb{N}}$ and
$(2^1 n + 1)_{n \in \mathbb{N}}$. This is the complete $k$-kernel, because all 
step sizes are of the form $k^l$. So taking larger step sizes will only yield 
the second or third sequence.

% Picture of mod-2 automaton in reverse reading. {{{1
\begin{graph}(0, 3)(-4, -1)
  \graphnodecolour{1}
  \graphnodesize{1}
  \roundnode{s1}(-1, 0) \nodetext{s1}(0, 0){$\iota$}
  \roundnode{s2}(-2, 1.5)
    \nodetext{s2}(0, -0.22){\circle{0.8}} \nodetext{s2}(0, 0){$a$}
  \roundnode{s3}(0, 1.5)
    \nodetext{s3}(0, -0.22){\circle{0.8}} \nodetext{s3}(0, 0){$b$}

  \diredge{s1}{s2} \edgetext{s1}{s2}{0}
  \diredge{s1}{s3} \edgetext{s1}{s3}{1}
  \dirloopedge{s2}{50}(-1, 0) \freetext(-3.6, 1.5){0,1}
  \dirloopedge{s3}{50}(1, 0) \freetext(1.6, 1.5){0,1}

  \freetext(-1, -0.7){$\Uparrow$}
\end{graph}
%}}}1
% Picture of mod-2 automaton in direct reading again. {{{1
\begin{graph}(0, 3)(-10, -2)
  \graphnodecolour{1}
  \graphnodesize{1}
  \roundnode{s1}(-2, 0) \nodetext{s1}(0, 0){$a$}
  \roundnode{s2}(0, 0)  \nodetext{s2}(0, 0){$b$}

  \dirloopedge{s1}{50}(-1, 0) \freetext(-3.6, 0){0}
  \dirbow{s1}{s2}{.2} \bowtext{s1}{s2}{.2}{1}
  \dirbow{s2}{s1}{.2} \bowtext{s2}{s1}{.2}{0}
  \dirloopedge{s2}{50}(1, 0) \freetext(1.6, 0){1}

  \freetext(-2, -0.7){$\Uparrow$}
\end{graph}
%}}}1

The left figure is the automaton that generates $u$ in reverse reading. Note 
that $\iota$ is not a final state, hence $\mathcal{S} \ne \mathcal{A}$. The 
final states are marked with an internal circle.

The right figure is the automaton that generates $u$ in direct reading. Hence
the generating automata can be essentially different.

\subsection{Equivalence between direct- and reverse reading}
\begin{theorem} \label{thm:direct_is_reverse}
A sequence is $k$-automatic in direct reading if and only if it is 
$k$-automatic in reverse reading.
\end{theorem}

\begin{proof}
Theorem \ref{thm:direct_reading} states that a $k$-automaton in direct reading 
is in bijection with a substitution of constant length. Theorem 
\ref{thm:reverse_reading} states that a $k$-automaton in reverse reading is in 
bijection with a finite $k$-kernel.

It remains to prove that a substitution of constant length results in a finite 
$k$-kernel.

$\Rightarrow$ Define $\sigma_i(a) : \mathcal{A \to A}$, which associates the
letter $a$ with the $i + 1$-th letter of its image in $\sigma$. We have 
$\sigma_i(u(n)) = u(kn + i)$, for any integer $n$ and for any 
$i \in \{0, \ldots, k - 1\}$. Let $l \ge 0$, $0 \le r \le k^l - 1$. Write
$r =  \sum_{i = 0}^{l - 1} r_i k^i$, where $0 \le r_i \le k - 1$. We thus have
$u(k^l n + r) = \sigma_{r_0}(\sigma_{r_1}( \ldots (\sigma_{r_{l - 1}}(u(n)))))$.
There are at most $|\mathcal{A}|^{|\mathcal{A}|}$ of these maps, so the
$k$-kernel is finite.

$\Leftarrow$ Suppose the $k$-kernel $N_k(u)$ of the sequence $u$ is finite.
Let $\overline{a}_1, \ldots, \overline{a}_d$ be the sequences in $N_k(u)$. Let
$U = (U(n))_{n \in \mathbb{N}}$ be the sequence with values in $\mathcal{A}^d$
defined by $U(n) = (\overline{a}_1(n), \ldots, \overline{a}_d(n))$. We 
construct a substitution $\sigma$ of constant length $k$ defined on 
$\mathcal{A}^d$ with $U$ as a fixed point. As $N_k(u)$ is stable by the maps
$\mathcal{A}_r$, where 
$\mathcal{A}_r(v(n))_{n \in \mathbb{N}} = (v(kn + r))_{n \in \mathbb{N}}$, then
for any $0 \le r \le k - 1$, $U(kn + r) = U(km + r)$, if $U(n) = U(m)$. We
define for $0 \le r \le k - 1$, $\sigma_r : \mathcal{A}^d \to \mathcal{A}^d$,
$U(n) \to U(nk +r)$ if there exists and $n$ such that 
$(a_1, \ldots, a_r) = U(n)$ and otherwise 
$(a_1, \ldots, a_r) \to (0, \ldots, 0)$. Hence the sequence $U$ is the fixed
point of the substitution of constant length 
$\sigma: a \to \sigma_0(a) \ldots \sigma_{k - 1}(a)$ and the sequence $u$ is
the image of $U$ by the projection of the first coordinate of $\mathcal{A}^d$ 
to $\mathcal{A}$.
\end{proof}

\begin{definition}
Because of Theorem \ref{thm:direct_is_reverse} we can define $k$-automaticity
as one of the following equivalences
\begin{itemize}
\item $k$-automaton in direct reading.
\item $k$-automaton in reverse reading.
\item a finite $k$-kernel. 
\item the fixed point of a substitution of constant length.
\end{itemize}
\end{definition}

\begin{corollary} \label{cor:homomorphism}
If we take a $k$-automatic sequence and apply some homomorphism to it, it will 
remain $k$-automatic.
\end{corollary}

\begin{proof}  
This is a direct result of Theorem \ref{thm:direct_reading} and Theorem
\ref{thm:direct_is_reverse}.
\end{proof}

\begin{example}\label{ex:homomorphism}\end{example}
% Picture of mod-2 automaton with exit map. {{{1
\begin{graph}(0, 3)(-4, -1.5)
  \graphnodecolour{1}
  \graphnodesize{1}
  \roundnode{s1}(-2, 0) \nodetext{s1}(0, 0){$a$/0}
  \roundnode{s2}(0, 0)  \nodetext{s2}(0, 0){$b$/1}

  \dirloopedge{s1}{50}(-1, 0) \freetext(-3.6, 0){0}
  \dirbow{s1}{s2}{.2} \bowtext{s1}{s2}{.2}{1}
  \dirbow{s2}{s1}{.2} \bowtext{s2}{s1}{.2}{0}
  \dirloopedge{s2}{50}(1, 0) \freetext(1.6, 0){1}

  \freetext(-2, -0.7){$\Uparrow$}
\end{graph}
%}}}1

This automaton has an output alphabet $Y$ different from $\mathcal{S}$. We 
have $Y = \{0, 1\}$ and we do not use Id$_{\{a, b\}}$ as the exit map, but
$\varphi(a) \to 0, \varphi(b) \to 1$. This is denoted by
$a/0, b/1$ in the automaton.

\section{Examples}
\subsection{The Prouhet-Thue-Morse sequence}
Let us look at the following substitution: $\sigma: \sigma(a) \to ab$,
$\sigma(b) \to ba$. With $a$ as the initial letter, it gives the
following fixed point:\\
\\
\monoit{u = abbabaabbaababbabaababbaabbabaabbaababbaabbabaababbab\ldots}\\
\\
The following 2-automaton is associated with this substitution:\\
% Picture of Prouhet-Thue-Morse automaton. {{{1
\begin{graph}(0, 3)(-4, -1.5)
  \graphnodecolour{1}
  \graphnodesize{1}
  \roundnode{s1}(-2, 0) \nodetext{s1}(0, 0){$a$}
  \roundnode{s2}(0, 0)  \nodetext{s2}(0, 0){$b$}

  \dirloopedge{s1}{50}(-1, 0) \freetext(-3.6, 0){0}
  \dirbow{s1}{s2}{.2} \bowtext{s1}{s2}{.2}{1}
  \dirbow{s2}{s1}{.2} \bowtext{s2}{s1}{.2}{1}
  \dirloopedge{s2}{50}(1, 0) \freetext(1.6, 0){0}

  \freetext(-2, -0.7){$\Uparrow$}
\end{graph}
%}}}1

Note that the final state is $a$ when the number of ones in the input word
is even, and $b$ otherwise. Hence this automaton generates $u$ both in direct
and in reverse reading.

The automaton above induces the following partitions of $\mathbb{N}$; 
\begin{eqnarray*}
\mathbb{N}_a &=& \{0, 3, 5, 6, 9, 10, 12, 15, \ldots\}\\
\mathbb{N}_b &=& \{1, 2, 4, 7, 8, 11, 13, 14, \ldots\}.
\end{eqnarray*}
Let us define $S_2(n)$ as the sum of the dyadic digits:
\begin{displaymath}
S_2(n) = \sum_{i \ge 0}n_i,\mathrm{\ if\ }
n = \sum_{i \ge 0}n_i2^i, n_i \in \{0, 1\}.
\end{displaymath}
It is obvious from the automaton that the sets $\mathbb{N}_a$ and 
$\mathbb{N}_b$ are defined by
\begin{eqnarray*}
x \in \mathbb{N}_a \Leftrightarrow S_2(n) \mathrm{\ is\ even},
x \in \mathbb{N}_b \Leftrightarrow S_2(n) \mathrm{\ is\ odd}.
\end{eqnarray*}

The 2-kernel of this sequence is\\
\\
\monoit{abbabaabbaababbabaababbaabbabaabbaababbaabbab\ldots} $(l = 0)$,\\
\monoit{baababbaabbabaababbabaabbaababbaabbabaabbaaba\ldots} $(l = 1, r = 1)$.\\
\\
So the 2-kernel consists of both fixed points.

\subsection{The Rudin-Shapiro sequence}
Define $u(n) = (-1)^{r_n}$, where $r_n$ is the number of occurrences of
consecutive `11' in the binary representation of $n$. It does not matter if we 
apply direct or reverse reading. The corresponding automaton is given by

% Picture of Rudin-Shapiro automaton. {{{1
\begin{graph}(0, 3)(-4, -1.5)
  \graphnodecolour{1}
  \graphnodesize{1}
  \roundnode{s1}(-2, 0) \nodetext{s1}(0, 0){\bs$a$/+1\es}
  \roundnode{s2}(0, 0)  \nodetext{s2}(0, 0){\bs$b$/+1\es}
  \roundnode{s3}(2, 0)  \nodetext{s3}(0, 0){\bs$c$/-1\es}
  \roundnode{s4}(4, 0)  \nodetext{s4}(0, 0){\bs$d$/-1\es}

  \dirloopedge{s1}{50}(-1, 0) \freetext(-3.6, 0){0}
  \dirbow{s1}{s2}{0.2} \bowtext{s1}{s2}{0.2}{1}
  \dirbow{s2}{s1}{0.2} \bowtext{s2}{s1}{0.2}{0}
  \dirbow{s2}{s3}{0.2} \bowtext{s2}{s3}{0.2}{1}
  \dirbow{s3}{s4}{0.2} \bowtext{s3}{s4}{0.2}{0}
  \dirbow{s3}{s2}{0.2} \bowtext{s3}{s2}{0.2}{1}
  \dirloopedge{s4}{50}(1, 0) \freetext(5.6, 0){0}
  \dirbow{s4}{s3}{0.2} \bowtext{s4}{s3}{0.2}{1}

  \freetext(-2, -0.7){$\Uparrow$}
\end{graph}
%}}}1

Thus the Rudin-Shapiro sequence is the invariant word under the substitution
$a \to ab, b \to ac, c \to db, d \to dc$
starting with $a$ and with the indicated output function.

\subsection{The Baum-Sweet sequence}
The Baum-Sweet sequence $(u_n)_{n \in \mathbb{N}}$ with values in the alphabet
$\{0, 1\}$ is defined by:

\vbox{\begin{eqnarray*}
u_n &=& 0 \mathrm{\ if\ the\ dyadic\ development\ of\ } n
          \mathrm{\ contains\ at\ least\ one\ odd\ string\ of\ 0's},\\
    &=& 1 \mathrm{\ if\ not.}
\end{eqnarray*}}
Obviously the automaton does not depend on the way of reading. It is given by

% Picture of Baum-Sweet automaton. {{{1
\begin{graph}(0, 4)(-4, -1.5)
  \graphnodecolour{1}
  \graphnodesize{1}
  \roundnode{s1}(-2, 0) \nodetext{s1}(0, 0){$a$/1}
  \roundnode{s2}(0, 0)  \nodetext{s2}(0, 0){$b$/1}
  \roundnode{s3}(2, 0)  \nodetext{s3}(0, 0){$c$/0}
  \roundnode{s4}(4, 0)  \nodetext{s4}(0, 0){$d$/0}

  \dirloopedge{s1}{50}(-1, 0) \freetext(-3.6, 0){0}
  \diredge{s1}{s2} \edgetext{s1}{s2}{1}
  \dirbow{s2}{s3}{0.2} \bowtext{s2}{s3}{0.2}{0}
  \dirloopedge{s2}{50}(0, 1) \freetext(0, 1.6){1}
  \dirbow{s3}{s2}{0.2} \bowtext{s3}{s2}{0.2}{0}
  \diredge{s3}{s4} \edgetext{s3}{s4}{1}
  \dirloopedge{s4}{50}(1, 0) \freetext(5.6, 0){0,1}

  \freetext(-2, -0.7){$\Uparrow$}
\end{graph}
%}}}1

It will be clear that the automaton above generates the sequence when we use
$\varphi(a) = \varphi(b) = 1$ and $\varphi(c) = \varphi(d) = 0$.

The automaton corresponds to the substitution $\sigma$ given by
$\sigma(a) \to ab, \sigma(b) \to cb$,
$\sigma(c) \to bd, \sigma(d) \to dd$.

\subsection{A divisibility automaton}
Given two integers $k$ and $d$ greater or equal to two, can an automaton 
decide only from its $k$-adic development whether any $n \in \mathbb{N}$ is
divisible by $d$? Let $\mathcal{S} = \{0, 1, \ldots, d - 1\}$ and let $u$ be the
periodic sequence:
\begin{displaymath}
  u = 01\ldots(d - 1)01\ldots(d - 1)\ldots
\end{displaymath}
We need to construct a $k$-automaton that generates $u$. To do this we need to
find a substitution $\sigma$ of constant length $k$ such that $u$ is the fixed
point of $\sigma$. We can do that by cutting $u$ in words of length $k$ and
rewriting $u$ as
$u = \sigma(0)\sigma(1)\ldots\sigma(d - 1)\sigma(0)\sigma(1)\ldots\sigma(d - 1)\ldots$

For the case $k = 2, d = 5$, the substitution looks as follows:

\vbox{\begin{eqnarray*}
  \sigma(0) &\to& 01\\
  \sigma(1) &\to& 23\\
  \sigma(2) &\to& 40\\
  \sigma(3) &\to& 12\\
  \sigma(4) &\to& 34
\end{eqnarray*}}
This gives the following automaton:\\
% Picture of mod-5 automaton. {{{1
\begin{graph}(0, 3)(-4, -0.5)
  \graphnodecolour{1}
  \graphnodesize{1}
  \roundnode{s1}(-2, 0) \nodetext{s1}(0, 0){$0$}
  \roundnode{s2}(-1, 1.5) \nodetext{s2}(0, 0){$1$}
  \roundnode{s3}(0, 0) \nodetext{s3}(0, 0){$2$}
  \roundnode{s4}(1, 1.5) \nodetext{s4}(0, 0){$3$}
  \roundnode{s5}(2, 0) \nodetext{s5}(0, 0){$4$}

  \dirloopedge{s1}{50}(-1, 0) \freetext(-3.6, 0){0}
  \diredge{s1}{s2} \edgetext{s1}{s2}{1}
  \diredge{s2}{s3} \edgetext{s2}{s3}{0}
  \dirbow{s2}{s4}{0.2} \bowtext{s2}{s4}{0.2}{1}
  \diredge{s3}{s5} \edgetext{s3}{s5}{0}
  \diredge{s3}{s1} \edgetext{s3}{s1}{1}
  \dirbow{s4}{s2}{0.2} \bowtext{s4}{s2}{0.2}{0}
  \diredge{s4}{s3} \edgetext{s4}{s3}{1}
  \diredge{s5}{s4} \edgetext{s5}{s4}{0}
  \dirloopedge{s5}{50}(1, 0) \freetext(3.6, 0){1}

  \freetext(-2, -0.7){$\Uparrow$}
\end{graph}\\
%}}}1
\\
Thus $n = (n_t n_{t - 1} \ldots n_0)_2$ is divisible by 5 if and only if 0 is
the final state after the consecutive transitions $n_t, n_{t - 1}, \ldots n_0$.
Moreover, if $j$ is the final state, then $j$ is the rest of $n$ after dividing
by 5.

\section{General automata}
\subsection{Regular languages}
\begin{definition}[Regular language] \label{def:regular_language}
A language $\mathcal{L}$ is called \emph{regular} if there exists a finite 
automaton that accepts $\mathcal{L}$. 
\end{definition}

Note that we do not talk about $k$-automata, but about automata in general.

\begin{lemma}[The pumping lemma for regular languages] \label{lem:pumping}
Let $\mathcal{L}$ be an infinite regular language. Then there exists an 
$n \in \mathbb{N}, n \ge 1$, such that for all $z \in \mathcal{L}$, if 
$|z| > n$ there are words $r, s, t$ such that
\begin{itemize}
\item $z = rst$
\item $s \ne \epsilon$
\item $|rs| \le n$
\item $r s^i t \in \mathcal{L}$ for all $i \in \mathbb{N}$ 
\end{itemize}
\end{lemma}

\begin{proof}
Let $A = (\mathcal{S}, \Delta, \delta, I, F)$ be a finite automaton which 
accepts $\mathcal{L}$. Let $K = \mathcal{L}(A)$, let $n = |\mathcal{S}|$, and 
let trans be a sequence of transitions. $\mathcal{S} \ne \epsilon$ because 
$\mathcal{L}$ is infinite, and therefore $n \ge 1$.

Consider $z \in K$ with $m = |z| > n$. If such a word does not exist, we are
finished. We look at the path $\pi$ in $A$ corresponding with $z$, say
$\pi = (q_0, a_1, q_1)(q_1, a_2, q_2)$ $\ldots(q_{m - 1}, a_m, q_m)$ with
$q_0 \in I$ and $q_m \in F$, so $z = \mathrm{trans}(\pi) = a_1a_2\ldots a_m$.
Because $m > n$, there exist $k$ and $j$ with $0 \le k < j \le n$ and
$q_k = q_j$. We now split the path into three parts; $\pi = \pi_1 \pi_2 \pi_3$,
with
\begin{eqnarray*}
  \pi_1 &=& (q_0, a_1, q_1) \ldots (q_{k - 1}, a_k, q_k),\\
  \pi_2 &=& (q_k, a_{k + 1}, q_{k + 1}) \ldots (q_{j - 1}, a_j, q_j),\\
  \pi_3 &=& (q_j, a_{j + 1}, q_{j + 1}) \ldots (q_{m - 1}, a_m, q_m).
\end{eqnarray*}
Let $r = \mathrm{trans}(\pi_1), s = \mathrm{trans}(\pi_2),
     t = \mathrm{trans}(\pi_3)$

Because $\pi_2$ is a cycle in the automaton, from $q_k$ to $q_j = q_k$, there
exists for all $i \in \mathbb{N}$ a path $\pi_1 \pi_2^i \pi_3$ from $q_0$ to
$q_m$. So $\mathrm{trans}(\pi_1 \pi_2^i \pi_3) \in K$. Because trans is a
homomorphism, we can say:  $\mathrm{trans}(\pi_1 \pi_2^i \pi_3) =
\mathrm{trans}(\pi_1) \mathrm{trans}(\pi_2)^i \mathrm{trans}(\pi_3) = r s^i t$.
So $r s^i t \in K$ for all $i \in \mathbb{N}$.

Note that $|s| = |\mathrm{trans}(\pi_2)| = j - k > 0$, so $s \ne \epsilon$,
and that $|rs| = |\mathrm{trans}(\pi_1 \pi_2)| = j \le n$. 
\end{proof}

% Picture of general automaton. {{{1
\begin{graph}(0, 3)(-4, -1.5)
  \graphnodecolour{1}
  \graphnodesize{1}
  \roundnode{s1}(-2, 0) \nodetext{s1}(0, 0){\bs$q_0$\es}
  \roundnode{s2}(0, 0) \nodetext{s2}(0, 0){\bS$q_k=q_j$\eS}
  \roundnode{s3}(2, 0)
    \nodetext{s3}(0, -0.22){\circle{0.8}} \nodetext{s3}(0, 0){\bs$q_m$\es}

  \diredge{s1}{s2} \edgetext{s1}{s2}{$r$}
  \dirloopedge{s2}{50}(0, 1) \freetext(0, 1.6){$s$}
  \diredge{s2}{s3} \edgetext{s2}{s3}{$t$}

  \freetext(-2, -0.7){$\Uparrow$}
\end{graph}
%}}}1

\begin{remark}
In a similar way, but by choosing $j$ maximal and $k$ minimal, we may prove 
the statement with the condition $|rs| \le n$ replaced with $|rt| \le n$.
\end{remark}

We can use the pumping lemma to prove that a certain language is not regular.

\begin{example} \label{ex:pumping}
$\mathcal{L} = \{0^m1^m | m \in \mathbb{N}\}$ is not regular.
\end{example}

\begin{proof}
Suppose $\mathcal{L}$ is regular. We can then find a number $n \ge 1$ that
conforms to the pumping lemma. Consider the word $z = 0^n 1^n$. We see that 
$z \in \mathcal{L}$ and $|z| = 2n > n$. So we can write $z = r s t$ with 
$|r s| \le n$ and $s \ne \epsilon$. Hence $rs$ consists of only zeros, so 
$s = 0^k$ for a certain $k$. Now take $i = 0$. According to the pumping lemma
$r s^0 t = r t \in \mathcal{L}$, but $r t = 0^{n - k} 1^n$ with $n - k < n$, so
$rt \notin \mathcal{L}$. This is a contradiction, so our assumption was false
and $\mathcal{L}$ is not regular. 
\end{proof}

\begin{theorem} \label{thm:not_regular}
If $\mathcal{L}(u)$ is regular and $u$ is minimal, then $u$ is periodic with
period $|s|$.
\end{theorem}

\begin{proof}
Suppose $\mathcal{L}(u)$ is regular. Let $z \in \mathcal{L}$. By Theorem 
\ref{lem:pumping}, we can write $z = r s t$ such that $r s^i t \in \mathcal{L}$ 
for every $i$. We have $|r s^i t|_1 = |r|_1 + i|s|_1 + |t|_1$. It follows that
$|s| |r s^i t|_1 - |s|_1 |r s^i t| = |s| |r|_1 + |s| |t|_1 - |r| |s|_1 - 
|t| |s|_1$ is independent of $i$. Hence $\lim_{i \to \infty} 
\frac{|r s^i t|_1}{|r s^i t|} - \frac{|s|_1}{|s|} = 0$. Since $r s^i t$ would
be a subword of $u$ for every $i$ and $u$ is minimal, it would follow from
Proposition 5.1.10 \cite{Fogg} page 105 that the frequency of 1 in $u$ is 
$\frac{|s|_1}{|s|} \in \mathbb{Q}$. 
\end{proof}

\begin{corollary} \label{cor:ptm_not_regular}
The language $\mathcal{L}(u)$ defined by the Prouhet-Thue-Morse sequence $u$ is
not regular.
\end{corollary}

\begin{proof} 
By Proposition 5.1.2 \cite{Fogg} page 102, $u$ is minimal and not periodic.
Theorem \ref{thm:not_regular} states that those sequences are not regular.
\end{proof}

\begin{corollary} \label{cor:sturmian_not_regular}
If $u$ is a Sturmian sequence, then $\mathcal{L}(u)$ is not regular.
\end{corollary}

\begin{proof} 
By Theorem 6.1.8 \cite{Fogg} a Sturmian sequence is minimal and by Theorem 
Proposition 6.1.10 \cite{Fogg} the frequency of 1 is irrational. Theorem 
\ref{thm:not_regular} states that those sequences are not regular.
\end{proof}

\subsection{The Fibonacci sequence}
We consider substitutions of non-constant length. They will result in 
non-complete automata, and therefore will only accept a subset of $\Delta^*$.\\
\\
The Fibonacci sequence is $u$ is the fixed point of the substitution 
$\sigma: \sigma(a) \to ab, \sigma(b) \to a$, hence\\
\\
\monoit{u = abaababaabaababaababaabaababaabaababaababaabaababaaba\ldots}\\
\\
The 2-kernel of this sequence contains the words:\\
\\
\monoit{abaababaabaababaababaabaababaabaababaababaabaababaaba\ldots}\\
\\
\monoit{aabbaabbaaabaaabaaabbaabbaaabaaabaaabba\ldots}\\
\monoit{baaabaaabbaabbaabbaaabaaabbaabbaabbaaab\ldots}\\
\\
\monoit{ababaaaaaabababababa\ldots}\\
\monoit{ababababababaaaaaaba\ldots}\\
\monoit{bababababaaaabababab\ldots}\\
\monoit{aaaabababababababaaa\ldots}\\
\monoit{}\\
\\
This suggests that the 2-kernel of this sequence is infinite. Its cardinality
certainly exceeds $|\mathcal{A}|^{|\mathcal{A}|}$, 
so the sequence is not 2-automatic and because this is the smallest 
substitution that generates the sequence, it is not $k$-automatic at all. The 
associated automaton is\\
% Picture of Fibonacci automaton. {{{1
\begin{graph}(0, 3)(-4, -1.5)
  \graphnodecolour{1}
  \graphnodesize{1}
  \roundnode{s1}(-2, 0) \nodetext{s1}(0, 0){$a$}
  \roundnode{s2}(0, 0)  \nodetext{s2}(0, 0){$b$}

  \dirloopedge{s1}{50}(-1, 0) \freetext(-3.6, 0){0}
  \dirbow{s1}{s2}{.2} \bowtext{s1}{s2}{.2}{1}
  \dirbow{s2}{s1}{.2} \bowtext{s2}{s1}{.2}{0}

  \freetext(-2, -0.7){$\Uparrow$}
\end{graph}\\
%}}}1
Note that this is almost the same automaton that generates $(ab)^{\mathbb{N}}$,
and that it only lacks one branch. We immediately see that this is not a 
$k$-automaton, for it is not total. Moreover, if we feed this automaton with 
the base 2 expansion of $n \in \mathbb{N}$, it will halt (or crash) when 
we feed it a string which has two consecutive ones as a substring (another
reason to conclude that this substitution is not $k$-automatic). However, our 
intuition is right, the automaton generates the Fibonacci sequence, but since 
it is not a $k$-automaton, we can not feed it with the base 2 expansion of all
the nonnegative integers. In this particular case we know which expansion to 
take, viz. the Fibonacci expansion of the nonnegative integers; in general this
is not known.

\begin{definition}[Zeckendorf expansion]
Let $(F_n)_{n \in \mathbb{N}}$ be the sequence of integers defined by $F_0 = 1,
F_1 = 2$ and for any integer $n > 1, F_{n + 1} = F_{n - 1} + F_n$.

If $n = \sum_{i = 0}^k n_i F_i$ with $n_k = 1, n_i \in \{0, 1\}$ and 
$\forall (i < k): n_i n_{i + 1} = 0$, we say that 
Fib$(n) = n_k n_{k - 1} \ldots n_0 \in \{0, 1\}^{k + 1}$ is the
\emph{Zeckendorf expansion} or \emph{Fibonacci representation} of the integer 
$n$.
\end{definition}

This gives the expansions\\ \vbox{
\begin{verbatim}
  0 = 0
  1 = 1
  2 = 10
  3 = 100
  4 = 101
  5 = 1000
   ...
\end{verbatim}}

\begin{theorem} \label{thm:fibonacci_numbersystem}
Every nonnegative integer $n$ can be written in a unique way as 
$n = \sum_{i \ge 0} n_i F_i$ with $n_i \in \{0, 1\}$ and
$\forall (i \ge 0): n_i n_{i + 1} = 0$
\end{theorem}

\begin{proof} By induction over $n$.\\
It is true for $n = 0$. Suppose it holds for $n < F_k$. If 
$F_k \le n < F_{k + 1}$ then by $F_{k + 1} = F_k + F_{k - 1}$ we have 
$n - F_k < F_{k - 1}$. Because we can write $n - F_k$ by our hypothesis as
$n - F_k = \sum_{i = 0}^{k - 2} n_i F_i$ with $n_i n_{i - 1} = 0$ for
$i = 1, \ldots, k - 2$, we can write $n = F_k + \sum_{i = 0}^{k - 2} n_i F_i$ 
such that there are no two consecutive ones.\\
\\
It remains to prove that the Fibonacci representation is unique.\\
Let $n \in \mathbb{N}$ be the smallest number for which there is more than one
representation. Choose $k$ such that $F_k \le n < F_{k + 1}$ and write
$n = n_k n_{k - 1} \ldots n_0$ (the standard representation) and
$n = n'_l n'_{l - 1} \ldots n'_0$ (another representation satisfying 
$n'_i n'_{i - 1} = 0$ for $i = 1, \ldots, k - 2$). Then $n_k = 1$ and $l \le k$.
If $n_k = n'_k$ then $n - F_k$ would have two distinct representations as well.
Hence $n'_k = 0$.
The maximum number of length $k - 1$ we can represent is 
$F_{k - 1} + F_{k - 3} + F_{k - 5} + \ldots$, which is equal to $F_k - 1$ and 
therefore smaller than $n$. 
\end{proof}

Since this system can be used to enumerate $n \in \mathbb{N}$ and it has the
property that no two ones succeed each other, this is the mapping we are
looking for.\\
Put
\begin{eqnarray*}
  \mathbb{N}_a &=& \{n \in \mathbb{N}, \mathrm{Fib}(n) \in \{0, 1\}^* 0\} = 
                   \{0, 2, 3, 5, 7, 8, 10, 11, 13, 15, \ldots\}\\
  \mathbb{N}_b &=& \{n \in \mathbb{N}, \mathrm{Fib}(n) \in \{0, 1\}^* 1\} =
                   \{1, 4, 6, 9, 12, 14, 17, 19, 22, 25, \ldots\}
\end{eqnarray*}
\\
Consider the first few substitutions of $\sigma$,\\
\\
$\sigma^0(a) =$ \monoit{a}\\
$\sigma^1(a) =$ \monoit{ab}\\
$\sigma^2(a) =$ \monoit{aba}\\
$\sigma^3(a) =$ \monoit{abaab}\\
$\sigma^4(a) =$ \monoit{abaababa}

\begin{theorem} \label{thm:fibonacci_fixedpoint}
$\sigma^n = \sigma^{n - 1} \sigma^{n - 2}, n \in \mathbb{N}, n \ge 2$
\end{theorem}

\begin{proof} By induction over $n$. Initial step:
$\sigma^2 = aba = \sigma^1 \sigma^0$.\\
Induction step: We have 
$\sigma^{n + 1} = \sigma(\sigma^n) = \sigma(\sigma^{n - 1} \sigma^{n - 2}) = 
\sigma(\sigma^{n - 1}) \sigma(\sigma^{n - 2}) = \sigma^n \sigma^{n - 1}$.
\end{proof}

We write in column $j$ the Fibonacci expansion of $j$ from above to below.\\
\\
\monoit{a b a ab aba abaab \ldots}\\
\verb#0 1 1 11 111 11111#\\
\verb#    0 00 000 00000#\\
\verb#      01 001 00011#\\
\verb#         010 00100#\\
\verb#             01001#\\
\\
We see that the Fibonacci expansions of the integers $n$ with 
$F_k \le n < F_{k + 1}$ are all of length $k$. We deduce from the definition of
the Fibonacci expansion strings that end with a 0 can be extended in two ways: 
$\ldots 0 \to \ldots 00, \ldots 01$, and strings that end with a 1 can only be 
extended in one way: $\ldots 1 \to \ldots 10$.

\begin{thebibliography}{XX}
\bibitem{Fogg} Fogg, N. Pytheas. Substitutions is Dynamics, Arithmetics and 
               Combinatorics, Springer Verlag, 2002.
\end{thebibliography}
\end{document}
