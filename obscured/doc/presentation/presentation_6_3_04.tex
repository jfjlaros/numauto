\documentclass{article}
\usepackage{/home/jeroen/studie/dw/graphs/graphs}
\usepackage{amsfonts, amssymb}
\frenchspacing
\begin{document}

\newcommand{\qed}{$\blacksquare$}

\section*{Automata}
Automatic sequences are connected in a fundamental way with substitutions of
\emph{constant length}.

\paragraph{Some definitions}
A finite automaton is a 5-tuple $A = (\mathcal{S}, \Delta, \delta, I, F)$
in which:\\
- $\mathcal{S}$: Finite set of states.\\
- $\Delta$: Finite alphabet of labels.\\
- $\delta \subseteq \mathcal{S} \times \Delta \times \mathcal{S}$:
Collection of transitions.\\
- $I \subseteq \mathcal{S}$: Collection of initial states.\\
- $F \subseteq \mathcal{S}$: Collection of final states.\\
\\
A finite automaton is represented by a directed graph with vertices
$\mathcal{S}$ called \emph{states}, edges $\delta$ called \emph{transitions}
and specially marked subsets of states $I$ and $F$, being the initial and final
states.\\
For our purposes, we will restrict ourselves to automata that have
$F = \mathcal{S}$ as the set of final states, so we will leave this notion
out unless stated otherwise. The states in $I$ are marked with a $\Uparrow$.\\
\\
An automaton is finite if $\mathcal{S}$ is finite.\\
\\
An automaton is deterministic if the following two conditions hold:\\
- $\exists (p \in I) \forall (q  \in \mathcal{S})
\{q = p \lor q \notin I\}$\\
- $\forall (p \in \mathcal{S}) \forall (a \in \Delta)
\forall (q \in \mathcal{S}) \forall (r \in \mathcal{S})
\{(p, a, q) \in \delta \land (p, a, r) \in \delta \Rightarrow q = r\}$.\\
In other words:\\
- There must be one and only one initial state.\\
- There can be no branches with the same label coming from the same state,
  going to different states.\\
\begin{graph}(0, 3)(-4, -1.5)
  \graphnodecolour{1}
  \graphnodesize{1}
  \roundnode{s1}(-2, 0) \nodetext{s1}(0, 0){$a$}
  \roundnode{s2}(0, 0) \nodetext{s2}(0, 0){$b$}

  \dirloopedge{s1}{50}(-1, 0) \freetext(-3.6, 0){0}
  \diredge{s1}{s2} \edgetext{s1}{s2}{0}

  \freetext(-2, -0.7){$\Uparrow$}
  \freetext(-2.5, -1.2){non-deterministic}
\end{graph}
\begin{graph}(0, 3)(-9, -1.5)
  \graphnodecolour{1}
  \graphnodesize{1}
  \roundnode{s1}(-2, 0) \nodetext{s1}(0, 0){$a$}
  \roundnode{s2}(0, 0) \nodetext{s2}(0, 0){$b$}

  \dirloopedge{s1}{50}(-1, 0) \freetext(-3.6, 0){0}
  \diredge{s1}{s2} \edgetext{s1}{s2}{1}

  \freetext(-2, -0.7){$\Uparrow$}
  \freetext(-2.5, -1.2){deterministic}
\end{graph}
\\
An automaton is complete (or total) if the following condition holds:\\
- $\forall (p \in \mathcal{S}) \forall (a \in \Delta)
\exists (q \in \mathcal{S}) \{(p, a, q) \in \delta\}$\\
In other words: Each state must have all branches from $\delta$.\\
\begin{graph}(0, 3)(-4, -1.5)
  \graphnodecolour{1}
  \graphnodesize{1}
  \roundnode{s1}(-2, 0) \nodetext{s1}(0, 0){$a$}
  \roundnode{s2}(0, 0) \nodetext{s2}(0, 0){$b$}

  \dirloopedge{s1}{50}(-1, 0) \freetext(-3.6, 0){0}
  \diredge{s1}{s2} \edgetext{s1}{s2}{1}

  \freetext(-2, -0.7){$\Uparrow$}
  \freetext(-2.5, -1.2){non-complete}
\end{graph}
\begin{graph}(0, 3)(-9, -1.5)
  \graphnodecolour{1}
  \graphnodesize{1}
  \roundnode{s1}(-2, 0) \nodetext{s1}(0, 0){$a$}
  \roundnode{s2}(0, 0) \nodetext{s2}(0, 0){$b$}

  \dirloopedge{s1}{50}(-1, 0) \freetext(-3.6, 0){0}
  \dirloopedge{s2}{50}(1, 0) \freetext(1.6, 0){0,1}
  \diredge{s1}{s2} \edgetext{s1}{s2}{1}

  \freetext(-2, -0.7){$\Uparrow$}
  \freetext(-2.5, -1.2){complete}
\end{graph}

\paragraph{The $k$-automaton}
The $k$-automaton (not a 2-tape automaton or transducer, as stated in the
book, but rather a Moore automaton) is a finite, deterministic and complete
automaton, expanded with an output function or exit map.\\
\\
Take $k \in \mathbb{N} \ge 2$.\\
$\mathcal{S}$: A finite set of states, there is a unique $i \in \mathcal{S}$
called the initial state.\\
$\Delta$: $k$ labels (integers from 0 to $k - 1$).\\
$\mathcal{L}$: The input language, in this case always $\Delta^*$.\\
$\delta$: $k$ transitions (per state) ($\cong \mathcal{S} \times \Delta$).\\
$Y$: The output alphabet. We usually take $Y = \mathcal{S}$.\\
$\varphi$: A function from $\mathcal{S}$ to $Y$ called the exit map. We
           usually take $\varphi =$ Id.\\
\\
So a $k$-automaton is an automaton with states $\mathcal{S}$. Each state has
$k$ outgoing branches, labelled $0, ..., k - 1$. Furthermore, there is only
one initial state. Each state also has an output function.

\paragraph{Example}
\begin{eqnarray*}
&&k = 2\\
&&\mathcal{S} = \{a, b\}, i \in \mathcal{S} = a\\
&&\Delta = \{0, 1\}\\
&&\sigma:\{a, b\} \rightarrow \{a, b\}^*, \sigma(a) \rightarrow ab,
\sigma(b) \rightarrow ab\\
&&Y = \{a, b\}\\
&&\varphi: \varphi(a) \rightarrow a, \varphi(b) \rightarrow b
\end{eqnarray*}
\\
This substitution (of constant length 2) has only one fixed point:\\
\\
\verb#  u = abababababababababababababababababababababababab....#
$= (ab)^\mathbb{N}$\\
\\
We can say: $u_n = a \Leftrightarrow n \mathrm{\ mod\ } 2 = 0$, and:
$u_n = b \Leftrightarrow n \mathrm{\ mod\ } 2 = 1$, so we need an automaton
that ends in state $a$ if $u_n$ is at an even position and in state $b$ if
$u_n$ is in an odd position.\\
\begin{graph}(0, 3)(-4, -1.5)
  \graphnodecolour{1}
  \graphnodesize{1}
  \roundnode{s1}(-2, 0) \nodetext{s1}(0, 0){$a$}
  \roundnode{s2}(0, 0)  \nodetext{s2}(0, 0){$b$}

  \dirloopedge{s1}{50}(-1, 0) \freetext(-3.6, 0){0}
  \dirbow{s1}{s2}{.2} \bowtext{s1}{s2}{.2}{1}
  \dirbow{s2}{s1}{.2} \bowtext{s2}{s1}{.2}{0}
  \dirloopedge{s2}{50}(1, 0) \freetext(1.6, 0){1}

  \freetext(-2, -0.7){$\Uparrow$}
\end{graph}\\
If we now take the base 2 expansion of an integer (let us say the decimal
number 22) and feed it to the automaton above, the automaton receives the
digits 10110 and will be in state $a$ when it starts.\\
After reading the first digit, it will be in state $b$, after reading the
second one the automaton will be in state $a$. The table below shows the path
taken:\\
\\
\begin{tabular}{c|c|c|l}
state & transition & next state & tail\\
\hline
$i=a$ & 1 & $b$ & 0110\\
$b$   & 0 & $a$ & 110\\
$a$   & 1 & $b$ & 10\\
$b$   & 1 & $b$ & 0\\
$b$   & 0 & $a$ &\\
$a$   &   &   &
\end{tabular}\\
\\
It will be clear that this automaton generates the string $(ab)^n$, for it
maps any integer $n$ to $a \Leftrightarrow n \mathrm{\ mod\ } 2 = 0$ and to
$b \Leftrightarrow n \mathrm{\ mod\ } 2 = 1$. This can be seen easily,
because if the automaton is in its last-but-one state and is about to read its
least significant digit, it will always reach state $a$ if this digit was an 0
and in state $b$ if it was an 1.\\
\\
If we now feed the automata the sequences 0, 1, 10, 11, 100, .... $\infty$,
we will get the sequence $u$. $\square$\\
\\
Question: Is is obvious or coincidence that this is the invariant word?\\
\\
A note about $\Delta$: Normally, an automaton accepts a string if it is in
a final state when the end of its input is read. In this case, the final state
results (via $\varphi$) in an output. The output will code a number
$n \in \mathbb{N}$ which will result in the $n$--th letter of the fixed point
if the base $k$ expansion of $n$ is fed to the automaton. Feeding the base
$k$ expansion of $n = 0, 1, 2, ...$ will result in the fixed point.

\paragraph{Automaticity} A sequence $(u_n)_{n \in \mathbb{N}}$ with values in
$Y$ is $k$-automatic if it can be generated by a $k$-automaton as follows:\\
- Let $\sum_{i=0}^j n_i k^i (n_j \neq 0)$ be the base $k$ expansion of an
  integer $n$.\\
- Initialise the automaton and feed it with the sequence $n_0, n_1, ..., n_j$,
  note that the least significant digit is read first.\\
- If the automaton has no more input it will be in state $a(n)$.\\
- Now put $u_n = \varphi(a(n))$.\\
\\
The definition above assumed the automaton was fed in \emph{reverse reading}.
We can also feed the digits to the automaton in reverse order (starting with
the most significant bit). This is called \emph{direct reading}.\\
Note that we use automata to generate a sequence, not to recognise one, as is
usual in computer science.

\paragraph{Letter-to-letter projection} Consider a map from a finite alphabet
$\mathcal{A}$ to an other finite alphabet $\mathcal{B}$. This map extends in
a natural way (by concatenation) to a map from
$\mathcal{A}^* \cup \mathcal{A}^\mathbb{N}$ to
$\mathcal{B}^* \cup \mathcal{B}^\mathbb{N}$.\\
\\
\paragraph{$k$-automaticity} A sequence $u$ is $k$-automatic if and only if
$u$ is the image by a letter-to-letter projection of a fixed point of a
substitution of constant length $k$.

\paragraph{Theorem} If we take an k-automatic sequence and apply some
homomorphism to it, it will stay k-automatic.

\paragraph{Proof} Let $v$ be the image of a letter-to-letter
projection $p: \mathcal{A} \rightarrow \mathcal{B}$ of the fixed point
$u = (u(n))_{n \in \mathbb{N}} \in \mathcal{A}^\mathbb{N}$ of a substitution
$\sigma$ of constant length $k$ defined on the alphabet $\mathcal{A}$, then
$v$ is generated by the same automaton that generated $u$, but with output
function the projection $p$. \qed

\paragraph{Example\\}
\begin{graph}(0, 3)(-4, -1.5)
  \graphnodecolour{1}
  \graphnodesize{1}
  \roundnode{s1}(-2, 0) \nodetext{s1}(0, 0){$a$/0}
  \roundnode{s2}(0, 0)  \nodetext{s2}(0, 0){$b$/1}

  \dirloopedge{s1}{50}(-1, 0) \freetext(-3.6, 0){0}
  \dirbow{s1}{s2}{.2} \bowtext{s1}{s2}{.2}{1}
  \dirbow{s2}{s1}{.2} \bowtext{s2}{s1}{.2}{0}
  \dirloopedge{s2}{50}(1, 0) \freetext(1.6, 0){1}

  \freetext(-2, -0.7){$\Uparrow$}
\end{graph}\\
This automaton has an output alphabet $Y$ different from $\mathcal{S}$. To
say: $Y = \{0, 1\}$ and we do not use Id$_{\{a, b\}}$ as the exit map, but
$\varphi(a) \rightarrow 0, \varphi(b) \rightarrow 1$. This is denoted by
$a/0, b/1$ in the automaton.

\paragraph{Theorem} A sequence $u$ is $k$-automatic in direct reading if and
only if $u$ is a fixed point of a substitution of constant length $k$.

\paragraph{Proof}
$\Leftarrow$ We have a fixed point $u$ of the substitution $\sigma$ of length
$k$ over the alphabet $\mathcal{A}$. Define
$\sigma_i: \mathcal{A} \rightarrow \mathcal{A}$, which sends letter $a$ to the
($i$ + 1)--th letter of $\sigma(a)$. We now have
$\sigma = \sigma_0 ... \sigma_{k - 1}$.\\
We can now construct an automaton in direct reading. Let $\mathcal{S = A}$ and
$\delta = \sigma_i$. There is a transition from $a$ to $b$ labelled $i$ if $b$
occurs in $\sigma(a)$ at position $i + 1$. Take $i \in \mathcal{S} = u_0$ as
the initial state. Take $\varphi =$ Id.\\
We have constructed an automaton that generates the sequence $u$ in direct
reading. Write $n = \sum_{i = 0}^j n_i k^i$, start from $u(0)$, then go to the
$(n_t + 1)$--th letter of $\sigma(u(0))$, denoted by $a_1$, then go to the
$(n_{t - 1} + 1)$--th letter of $\sigma(a_1)$, which is also the
$(kn_t + n_{t - 1} + 1)$--th letter of $\sigma^2(u(0))$ and after $n$ steps, we
arrive at the $(n + 1)$--th letter of $\sigma^{t + 1}(u_0)$, which is $u(n)$.\\
$\Rightarrow$ Let $u$ be a sequence generated by a $k$-automaton in direct
reading. Let $\mathcal{S}$ be the set of states of the automaton and let
$f_0, ..., f_{k - 1}$ be the transition maps (i.e. the state we reach if we
follow transition $0, ..., k$ from a certain state). Define the substitution of
constant length $\sigma = f_0, ..., f_{k - 1}$ over $\mathcal{S}$. Let $v$ be
the fixed point of $\sigma$ beginning with the initial state $i$. It is easily
checked that the sequence $u$ is the image by the output function $\varphi$ of
the fixed point $v$. \qed

\section*{The Cantor sequence}
The Cantor sequence is defined as follows: $\sigma: \sigma(a) \rightarrow aba,
\sigma(b) \rightarrow bbb$. With $a$ as the initial letter, it gives the
following fixed point:
\begin{verbatim}
  u = ababbbababbbbbbbbbababbbababbbbbbbbbbbbbbbbbbbbbbbbbb...
\end{verbatim}
The following automaton is associated with this
substitution:\\
\begin{graph}(0, 3)(-4, -1.5)
  \graphnodecolour{1}
  \graphnodesize{1}
  \roundnode{s1}(-2, 0) \nodetext{s1}(0, 0){$a$}
  \roundnode{s2}(0, 0)  \nodetext{s2}(0, 0){$b$}

  \dirloopedge{s1}{50}(-1, 0) \freetext(-3.6, 0){0,2}
  \diredge{s1}{s2} \edgetext{s1}{s2}{1}
  \dirloopedge{s2}{50}(1, 0) \freetext(1.6, 0){0,1,2}

  \freetext(-2, -0.7){$\Uparrow$}
\end{graph}\\
Let us denote $\mathbb{C}_a$ as the set of integers $n$ such that the
$(n + 1)$--th letter of the Cantor sequence is $a$.\\
The automaton above suggests the following mapping from
$\mathbb{N}$ to $u(n)$:
\begin{displaymath}
  \mathbb{C}_a = \Big\{n \in \mathbb{N}; n = \sum_{i \ge 0} n_i3^i,
  \mathrm{\ with\ } \forall(i \ge 0), n_i \in \{0, 2\}\Big\}
\end{displaymath}

\paragraph{Proof} Consider the Cantor fractal:
\begin{verbatim}
   0      0.1       0.2      1  |   0     0.1         0.2     1
---------------------------------------------------------------
   ===========================  |               a
   =========         =========  |      a        b        a
   ===   ===         ===   ===  |   a  b  a  b  b  b  a  b  a
   = =   = =         = =   = =  |  ababbbababbbbbbbbbababbbaba
\end{verbatim}
If we now write all number in base 3, and replace 0.1 with 0.0$\overline2$
(in fact we write all numbers ending with an 1 like this) we are left with
all numbers between 0 and 1 that can be written in ternary notation with no
'1' in any position. This is ${\mathbb{C}_a}^{-1}$ (by $^{-1}$ we mean that
each element in $\mathbb{C}_a$ must be raised to the power of $-1$ to obtain
the corresponding element in ${\mathbb{C}_a}^{-1}$).\\
So by taking the inverse of this set, we get $\mathbb{C}_a$. \qed

\section*{The kernel of a sequence}
\paragraph{$k$-kernel} Let $N_k(u)$ be the set of subsequences of the sequence
$(u(n))_{n \in \mathbb{N}}$ defined by:
\begin{displaymath}
N_k(u) = \{(u(k^ln + r))_{n \in \mathbb{N}}; l \ge 0; 0 \le r \le k^l - 1\}.
\end{displaymath}

\paragraph{Theorem} A sequence $u \in \mathcal{A}^\mathbb{N}$ is $k$-automatic
in reverse reading if and only if the $k$-kernel $N_k(u)$ of the sequence $u$
is finite.

\paragraph{Proof}
$\Leftarrow$ Suppose the $k$-kernel of a sequence $u$ is finite. Let
$u = u_1, ..., u_d$ be the sequences of $N_k(u)$. Take
$\mathcal{S} = \{a_1, ..., a_d\}$. Define for any $r \in [0, k - 1]$, the map
$r : \mathcal{S} \rightarrow \mathcal{S}$ which associates any state $a_i$ with
$r.a_i$ in bijection with the sequence of the $k$-kernel
$(u_i(kn + r))_{n \in \mathbb{N}}$. So $r.a_i$ is the state which will be
reached if we follow label $r$ from state $a_i$. Let
$n = \sum_{i = 0}^j n_i k^i$ be the base $k$ expansion of an integer $n$, with
$n_i \in [0, k - 1], n_j \ne 0$.  We define the map $n$ from
$\mathcal{S}$ to $\mathcal{S}$ by $n(a_i) = n_j(n_{j - 1}( ... (n_0(a_i))))$,
if $n \ne 0$, otherwise the map 0 is the identity.\\
We see that $n.a_i$ is in bijection with
$(u(k^{j + 1} l + n))_{l \in \mathbb{N}}$. Hence if $n.a_1 = m.a_1$. then
$u(n) = u(m)$, by considering the first term of the two corresponding
subsequences. Now we can define the map $\varphi$ as output function, which
associates $a_i$ with $u(n)$ such that $n.a_1 = a_i$. Therefore the sequence
is generated by the automaton in reverse reading.\\
$\Rightarrow$ Suppose $A$ is a finite $k$-automaton with initial state $i$
which generates $u$. The subsequence $(u(k^ln + r))_{n \ge 0}$, where
$l \ge 0$ and $0 \le r < k^l$, is generated by $A$ with the initial state
$\overline{r}.i$, where $\overline{r}$ is the word of $k$ letters obtained
by concatenating in front of the base $k$ expansion of $r$ as many zeros as
necessary. Because $A$ has a finite number of states, there is only a finite
set of subsequences. \qed\\
\\
The proof above gives a method of constructing a $k$-automaton in reverse
reading. Take the $k$-kernel of a fixed point and let $\mathcal{S}$ be in
bijection with the sequences. Now add transitions from state $a$ to $b$
labelled $i$ if the sequence associated with $b$ is a subsequence of the
sequence associated with $a$ defined as $(u_i(kn + r))_{n \in \mathbb{N}}$.
Make one of the fixed points the initial state and make all states that have
incoming transitions a final state.

\paragraph{Example}
The fixed point of the substitution $a \rightarrow ab$, $b \rightarrow ab$,
which is $abababab...$ has the following 2-kernel.\\
\\
\verb#  ababababababababababababababababababa...#
$= (ab)^\mathbb{N} = (u(2^0 n + 0))_{n \in \mathbb{N}}$\\
\verb#  aaaaaaaaaaaaaaaaaaaaaaaaaaaaaaaaaaaaa...#
$= a^\mathbb{N} = (u(2^1 n + 0))_{n \in \mathbb{N}}$\\
\verb#  bbbbbbbbbbbbbbbbbbbbbbbbbbbbbbbbbbbbb...#
$= b^\mathbb{N} = (u(2^1 n + 1))_{n \in \mathbb{N}}$\\
\\
By taking $l = 0$, we find $(2^0 n + 0)_{n \in \mathbb{N}}$. $l = 1$ Gives
two sequences: $(2_1 n + 0)_{n \in \mathbb{N}}$ and
$(2_1 n + 1)_{n \in \mathbb{N}}$.\\
This is the complete $k$-kernel, because all step sizes are of the form
$k^l$. So taking larger step sizes will only yield the second or third
sequence again.\\
\begin{graph}(0, 3)(-4, -0.5)
  \graphnodecolour{1}
  \graphnodesize{1}
  \roundnode{s1}(-1, 0) \nodetext{s1}(0, 0){$i$}
  \roundnode{s2}(-2, 1.5)
    \nodetext{s2}(0, -0.22){\circle{0.8}} \nodetext{s2}(0, 0){$a$}
  \roundnode{s3}(0, 1.5)
    \nodetext{s3}(0, -0.22){\circle{0.8}} \nodetext{s3}(0, 0){$b$}

  \diredge{s1}{s2} \edgetext{s1}{s2}{0}
  \diredge{s1}{s3} \edgetext{s1}{s3}{1}
  \dirloopedge{s2}{50}(-1, 0) \freetext(-3.6, 1.5){0,1}
  \dirloopedge{s3}{50}(1, 0) \freetext(1.6, 1.5){0,1}

  \freetext(-1, -0.7){$\Uparrow$}
\end{graph}\\
\\
This is the automaton that generates $u$ in reverse reading. Note that since
$\mathcal{A \ne S}$ not all states are final states. To say: $i$ is not a
final state. The states that are final states are marked with an internal
circle.

\paragraph{Theorem} A sequence is $k$-automatic in direct reading if and only
if it is $k$-automatic in reverse reading.

\paragraph{Proof}
From the proofs above, we can conclude that we can construct a $k$-automaton in
direct reading from a substitution of constant length, which generates $u$,
which in turn defines the $k$-kernel, from which we can construct an automaton
in reverse reading. This proof also holds the other way around. \qed

\section*{The Thue-Morse sequence}
Let us look at the following substitution: $\sigma: \sigma(a) \rightarrow ab$,
$\sigma(b) \rightarrow ba$. With $a$ as the initial letter, it gives the
following fixed point:
\begin{verbatim}
  u = abbabaabbaababbabaababbaabbabaabbaababbaabbabaababbab...
\end{verbatim}
The following automaton is associated with this
substitution:\\
\begin{graph}(0, 3)(-4, -1.5)
  \graphnodecolour{1}
  \graphnodesize{1}
  \roundnode{s1}(-2, 0) \nodetext{s1}(0, 0){$a$}
  \roundnode{s2}(0, 0)  \nodetext{s2}(0, 0){$b$}

  \dirloopedge{s1}{50}(-1, 0) \freetext(-3.6, 0){0}
  \dirbow{s1}{s2}{.2} \bowtext{s1}{s2}{.2}{1}
  \dirbow{s2}{s1}{.2} \bowtext{s2}{s1}{.2}{1}
  \dirloopedge{s2}{50}(1, 0) \freetext(1.6, 0){0}

  \freetext(-2, -0.7){$\Uparrow$}
\end{graph}\\
Note that this automaton generates $u$ in either direct or reverse reading.

\paragraph{Theorem} The above automaton generates the Thue-Morse sequence.

\paragraph{Proof} Define bin($n$) as the dyadic development of the integers and
ones($n$) as the number of ones modulo 2 in the base 2 expansion of the integer
$n$.\\
$\forall (n \ge 1)$ \{bin($2n$) = bin$(n) \cdot 0$ and bin($2n + 1$) =
bin$(n) \cdot 1$\}\\
So $\forall (n \in \mathbb{N})$ \{ones($n$) = ones$(2n) \ne$ ones$(2n + 1)$\}\\
The same pattern occurs in the Thue-Morse sequence, which is the fixed point
of substitution $\sigma$.
\begin{eqnarray*}
u_n = a &\Rightarrow& u_{2n} \cdot u_{2n + 1} = ab\\
u_n = b &\Rightarrow& u_{2n} \cdot u_{2n + 1} = ba
\end{eqnarray*}
We see a 1-1 mapping from ones($n$) to the Thue-Morse sequence. \qed\\
\\
A more intuitive argument:\\
The automaton above suggests an interesting mapping from $\mathbb{N}$ to
$u(n)$. To say:
\begin{eqnarray*}
\mathbb{N}_a = \{0, 3, 5, 6, 9, 10, 12, 15, ...\}\\
\mathbb{N}_b = \{1, 2, 4, 7, 8, 11, 13, 14, ...\}
\end{eqnarray*}
Let us define $S_2(n)$ as the sum of the dyadic digits:
\begin{displaymath}
S_2(n) = \sum_{i \ge 0}n_i,\mathrm{\ if\ }
n = \sum_{i \ge 0}n_i2^i, n_i \in \{0, 1\}
\end{displaymath}
The two disjunct sets $\mathbb{N}_a$ and $\mathbb{N}_b$ are defined by:
\begin{eqnarray*}
x \in \mathbb{N}_a &\Leftrightarrow& S_2(n) \mathrm{\ is\ even}\\
x \in \mathbb{N}_b &\Leftrightarrow& S_2(n) \mathrm{\ is\ odd}
\end{eqnarray*}
This can be seen from the automaton. It will start in position $a$,
which is an even position, it will be in state $b$ if it has read an odd
amount of ones and it will be in state $a$ if it has read an even amount
of ones.\\
The 2-kernel of this sequence is:\\
\\
\verb#  abbabaabbaababbabaababbaabbabaabbaababbaabbab...# $(l = 0)$\\
\verb#  baababbaabbabaababbabaabbaababbaabbabaabbaaba...# $(l = 1, r = 1)$\\
\\
So, only the fixed points are part of this 2-kernel.

\section*{The Rudin-Shapiro sequence}
Define $u(n) = (-1)^{r_n}$, where $r_n$ is the number of occurrences of
consecutive `11' in the binary representation of $n$.

\begin{graph}(0, 3)(-4, -1.5)
  \graphnodecolour{1}
  \graphnodesize{1}
  \roundnode{s1}(-2, 0) \nodetext{s1}(0, 0){$a$/+1}
  \roundnode{s2}(0, 0)  \nodetext{s2}(0, 0){$b$/+1}
  \roundnode{s3}(2, 0)  \nodetext{s3}(0, 0){$c$/-1}
  \roundnode{s4}(4, 0)  \nodetext{s4}(0, 0){$d$/-1}

  \dirloopedge{s1}{50}(-1, 0) \freetext(-3.6, 0){0}
  \dirbow{s1}{s2}{0.2} \bowtext{s1}{s2}{0.2}{1}
  \dirbow{s2}{s1}{0.2} \bowtext{s2}{s1}{0.2}{0}
  \dirbow{s2}{s3}{0.2} \bowtext{s2}{s3}{0.2}{1}
  \dirbow{s3}{s4}{0.2} \bowtext{s3}{s4}{0.2}{0}
  \dirbow{s3}{s2}{0.2} \bowtext{s3}{s2}{0.2}{1}
  \dirloopedge{s4}{50}(1, 0) \freetext(5.6, 0){0}
  \dirbow{s4}{s3}{0.2} \bowtext{s4}{s3}{0.2}{1}

  \freetext(-2, -0.7){$\Uparrow$}
\end{graph}\\

\section*{The Baum-Sweet sequence}
The Baum-Sweet sequence $(u_n)_{n \in \mathbb{N}}$ with values in the alphabet
$\{0, 1\}$ is defined by:
\begin{eqnarray*}
u_n &=& 0 \mathrm{\ if\ the\ dyadic\ development\ of\ } n
          \mathrm{\ contains\ at\ least\ one\ odd\ string\ of\ 0's},\\
    &=& 1 \mathrm{\ if\ not.}
\end{eqnarray*}
\begin{graph}(0, 3)(-4, -1.5)
  \graphnodecolour{1}
  \graphnodesize{1}
  \roundnode{s1}(-2, 0) \nodetext{s1}(0, 0){$a$/1}
  \roundnode{s2}(0, 0)  \nodetext{s2}(0, 0){$b$/1}
  \roundnode{s3}(2, 0)  \nodetext{s3}(0, 0){$c$/0}
  \roundnode{s4}(4, 0)  \nodetext{s4}(0, 0){$d$/0}

  \dirloopedge{s1}{50}(-1, 0) \freetext(-3.6, 0){0}
  \diredge{s1}{s2} \edgetext{s1}{s2}{1}
  \dirbow{s2}{s3}{0.2} \bowtext{s2}{s3}{0.2}{0}
  \dirloopedge{s2}{50}(0, 1) \freetext(0, 1.6){1}
  \dirbow{s3}{s2}{0.2} \bowtext{s3}{s2}{0.2}{0}
  \diredge{s3}{s4} \edgetext{s3}{s4}{1}
  \dirloopedge{s4}{50}(1, 0) \freetext(5.6, 0){0,1}

  \freetext(-2, -0.7){$\Uparrow$}
\end{graph}\\
It will be clear that the automaton above generates the sequence when we use
$\varphi(a) = \varphi(b) = 1$ and $\varphi(c) = \varphi(d) = 0$.\\
The automaton in turn gives the substitution rules:
$\sigma(a) \rightarrow ab, \sigma(b) \rightarrow cb$,
$\sigma(c) \rightarrow bd, \sigma(d) \rightarrow dd$.

\section*{A divisibility automaton}
Given two integers $k$ and $d$ greater or equal to two, is it possible to
decide only from its $k$-adic development whether any $n \in \mathbb{N}$ is
divisible by $d$? Let $\mathcal{S} = \{0, 1, ..., d - 1\}$ and let $u$ be the
periodic sequence:
\begin{displaymath}
  u = 01...(d - 1)01...(d - 1)... .
\end{displaymath}
We need to construct an automaton that generates $u$. To do this we need to
find a substitution $\sigma$ such that $u$ is the fixed point of $\sigma$.
We can do that by cutting $u$ in words of length $k$ and rewriting $u$ as
$u = \sigma(0)\sigma(1)...\sigma(d - 1)\sigma(0)\sigma(1)...\sigma(d - 1)...$.\\
\\
For the case $k = 2, d = 5$, the substitution looks as follows:
\begin{eqnarray*}
  \sigma(0) &\rightarrow& 01\\
  \sigma(1) &\rightarrow& 23\\
  \sigma(2) &\rightarrow& 40\\
  \sigma(3) &\rightarrow& 12\\
  \sigma(4) &\rightarrow& 34
\end{eqnarray*}
This gives the following automaton:\\
\begin{graph}(0, 3)(-4, -0.5)
  \graphnodecolour{1}
  \graphnodesize{1}
  \roundnode{s1}(-2, 0) \nodetext{s1}(0, 0){$0$}
  \roundnode{s2}(-1, 1.5) \nodetext{s2}(0, 0){$1$}
  \roundnode{s3}(0, 0) \nodetext{s3}(0, 0){$2$}
  \roundnode{s4}(1, 1.5) \nodetext{s4}(0, 0){$3$}
  \roundnode{s5}(2, 0) \nodetext{s5}(0, 0){$4$}

  \dirloopedge{s1}{50}(-1, 0) \freetext(-3.6, 0){0}
  \diredge{s1}{s2} \edgetext{s1}{s2}{1}
  \diredge{s2}{s3} \edgetext{s2}{s3}{0}
  \dirbow{s2}{s4}{0.2} \bowtext{s2}{s4}{0.2}{1}
  \diredge{s3}{s5} \edgetext{s3}{s5}{0}
  \diredge{s3}{s1} \edgetext{s3}{s1}{1}
  \dirbow{s4}{s2}{0.2} \bowtext{s4}{s2}{0.2}{0}
  \diredge{s4}{s3} \edgetext{s4}{s3}{1}
  \diredge{s5}{s4} \edgetext{s5}{s4}{0}
  \dirloopedge{s5}{50}(1, 0) \freetext(3.6, 0){1}

  \freetext(-2, -0.7){$\Uparrow$}
\end{graph}\\

\section*{Sturmian sequences}
\paragraph{Regular languages}
A language $\mathcal{L}$ is regular if and only if we can make a finite
automaton that accepts $\mathcal{L}$. Note that we do not talk about
$k$-automata, but about automata in general.

\paragraph{The pumping lemma for regular languages}
\paragraph{Theorem}
Let $\mathcal{L}$ be a regular language. Then there exists an
$n \in \mathbb{N}, n \ge 1$, such that for all $z \in \mathcal{L}$, if
$|z| > n$ there are words $u, v, w$ such that:\\
- $z = uvw$\\
- $v \ne \epsilon$\\
- $|uv| \le n$\\
- For all $i \in \mathbb{N}, uv^iw \in \mathcal{L}$

\paragraph{Proof}
Let $A = (\mathcal{S}, \Delta, \delta, I, F)$ be a finite automaton with
$K = \mathcal{L}(A)$ and let $n = |\mathcal{S}|$. We can assume
$\mathcal{S} \ne \epsilon$, and therefore $n \ge 1$.\\
Take any $z \in K$ with $|z| > n$. If such a word does not exist, we are
finished. We look at a path $\pi$ before $z$ in $A$, say
$\pi = (q_0, a_1, q_1)(q_1, a_2, q_2)...(q_{m - 1}, a_m, q_m)$ with
$q_0 \in I$ and $q_m in F$, so $z = \mathrm{trans}(\pi) = a_1a_2...a_m$.\\
Because $m > n$, there exist a $k$ and $j$ where $0 \le k < j \le n$ and
$q_k = q+j$.\\
We now spit the path in three parts: $\pi = \pi_1 \pi_2 \pi_3$ with
\begin{eqnarray*}
  \pi_1 &=& (q_0, a_1, q_1) ... (q_{k - 1}, a_k, q_k),\\
  \pi_2 &=& (q_k, a_{k + 1}, q_{k + 1}) ... (q_{j - 1}, a_j, q_j),\\
  \pi_3 &=& (q_j, a_{j + 1}, q_{j + 1}) ... (q_{m - 1}, a_m, q_m).
\end{eqnarray*}
Now choose $u = \mathrm{trans}(\pi_1), v = \mathrm{trans}(\pi_2),
            w = \mathrm{trans}(\pi_3)$\\
Because $\pi_2$ is a cycle in the automaton, from $q_k$ to $q_j = q_k$, there
exists for all $i \in \mathbb{N}$ a path $\pi_1 \pi_2^i \pi_3$ from $q_0$ to
$q_m$. So $\mathrm{trans}(\pi_1 \pi_2^i \pi_3) \in K$. Because trans is a
homomorphism, we can say:  $\mathrm{trans}(\pi_1 \pi_2^i \pi_3) =
\mathrm{trans}(\pi_1) \mathrm{trans}(\pi_2)^i \mathrm{trans}(\pi_3) = uv^iw$.
So $uv^iw \in K$.\\
Note that $|v| = |\mathrm{trans}(\pi_2)| = j - k > 0$, so $v \ne \epsilon$
and also $|uv| = |\mathrm{trans}(\pi_1 \pi_2| = j \le n$. \qed\\
\begin{graph}(0, 3)(-4, -1.5)
  \graphnodecolour{1}
  \graphnodesize{1}
  \roundnode{s1}(-2, 0) \nodetext{s1}(0, 0){$q_0$}
  \roundnode{s2}(0, 0) \nodetext{s2}(0, 0){$q_k = q_j$}
  \roundnode{s3}(2, 0)
    \nodetext{s3}(0, -0.22){\circle{0.8}} \nodetext{s3}(0, 0){$q_m$}

  \diredge{s1}{s2} \edgetext{s1}{s2}{$u$}
  \dirloopedge{s2}{50}(0, 1) \freetext(0, 1.6){$v$}
  \diredge{s2}{s3} \edgetext{s2}{s3}{$w$}

  \freetext(-2, -0.7){$\Uparrow$}
\end{graph}\\
This lemma says something about the structure of languages. If a certain
language does not conform to this lemma, it is not regular. Note that we can
only use this lemma to prove a certain language is not regular.

\paragraph{Theorem}
$\mathcal{L} = \{0^m1^m | m \in \mathbb{N}\}$ Is not regular.

\paragraph{Proof}
Suppose $\mathcal{L}$ is regular. We can then find a number $\ge 1$ that
conforms to the pumping lemma.\\
Consider the word $z = 0^n1^n$. We see that $z \in \mathcal{L}$ and
$|z| = 2n > n$. So we can write $z = uvw$. Because $|uv| \le n$, $uv$ only
consists of zeros, so $v = 0^k$ for a certain $k$. Now take $i = 0$, according
to the pumping lemma $uv^0w = uw \in \mathcal{L}$, but $uw = 0^{n - k}1^n$
with $n - k < n$, so $uw \notin \mathcal{L}$. This is a contradiction, so our
assumption was false. $\mathcal{L}$ is not regular. \qed

\paragraph{Theorem}
If $u$ is a Sturmian sequence, the language $\mathcal{L}(u)$ is not regular.

\paragraph{Proof}
Suppose $\mathcal{L}(u)$ is regular, then for any $i$ $uv^iw$ belongs to
$\mathcal{L}$. So $v^i$ belongs to $\mathcal{L}$. The frequency of 1 in $v^i$
is $|v|_1 / |v|$, which is $\in \mathbb{Q}$. This is in contradiction with
proposition 6.1.10. \qed

\section*{The Fibonacci sequence}
Up to now, we have only looked at substitutions of constant length, which had
the important property that the associated automaton accepted
$\Delta^*$. Now let us look at substitutions of non-constant
length. They will result in non-complete automata, and therefore will only
accept a subset of $\Delta^*$.\\
\\
The Fibonacci sequence is as follows: $\sigma: \sigma(a) \rightarrow ab,
\sigma(b) \rightarrow a$. With $a$ as the initial letter, it gives the
following fixed point:
\begin{verbatim}
  u = abaababaabaababaababaabaababaabaababaababaabaababaaba...
\end{verbatim}
The 2-kernel of this sequence is not finite:
\begin{verbatim}
  abaababaabaababaababaabaababaabaababaababaabaababaaba...

  aabbaabbaaabaaabaaabbaabbaaabaaabaaabbaa...
  baaabaaabbaabbaabbaaabaaabbaabbaabbaaab...

  ababaaaaaabababababa...
  ababababababaaaaaaba...
  bababababaaaabababab...
  aaaabababababababaaa...
  ...
\end{verbatim}
This suggests that the 2-kernel of this sequence is infinite. It is at least
bigger than $(\mathrm{Card}(\mathcal{A}))^{\mathrm{Card}(\mathcal{A})}$, so
this sequence is not 2-automatic.\\
\\
This substitution suggests the following automaton is associated with this
substitution:\\
\begin{graph}(0, 3)(-4, -1.5)
  \graphnodecolour{1}
  \graphnodesize{1}
  \roundnode{s1}(-2, 0) \nodetext{s1}(0, 0){$a$}
  \roundnode{s2}(0, 0)  \nodetext{s2}(0, 0){$b$}

  \dirloopedge{s1}{50}(-1, 0) \freetext(-3.6, 0){0}
  \dirbow{s1}{s2}{.2} \bowtext{s1}{s2}{.2}{1}
  \dirbow{s2}{s1}{.2} \bowtext{s2}{s1}{.2}{0}

  \freetext(-2, -0.7){$\Uparrow$}
\end{graph}\\
Note that this is almost the same automaton that generates
$(ab)^{\mathbb{N}}$, it only lacks one branch.\\
We immediately see that this is not a $k$-automaton, for it is not total,
moreover, if we feed this automaton with the base 2 expansion of
$n \in \mathbb{N}$, it will halt (or crash) when we feed it a string which has
two consecutive ones as a substring.\\
However, our intuition is right, this is the automaton associated with the
Fibonacci sequence, but since it is not a $k$-automaton, we can not feed it
with the base $k$ expansion of an integer. We must find an other type of
expansion.\\
In this particular case we know what kind of expansion to take; in general this
is not known.\\
\\
Let $(F_n)_{n \in \mathbb{N}}$ be the sequence of integers defined by $F_0 = 1,
F_1 = 2$ and for any integer $n > 1, F_{n + 1} = F_{n - 1} + F_n$.\\
\\
The Fibonacci expansion: If $n = \sum_{i = 0}^k n_i F_i$ with
$n_k = 1, n_i \in \{0, 1\}$ and $\forall (i < k) \{n_i n_{i + 1} = 0\}$, we
say that Fib$(n) = n_k n_{k - 1} ... n_0 \in \{0, 1\}^{k + 1}$ is the
Fibonacci expansion of the integer $n$.\\
\\
This gives the following mapping:
\begin{verbatim}
  0 = 0
  1 = 1
  2 = 10
  3 = 100
  4 = 101
  5 = 1000
   ...
\end{verbatim}

\paragraph{Theorem}
Every nonnegative integer $n$ can be written in a unique way as
$n = \sum_{i \ge 0} n_i F_i$ with $n_i \in \{0, 1\}$ and
$\forall (i \ge 0) \{n_i n_{i + 1} = 0\}$

\paragraph{Proof} By induction:\\
Suppose this holds for $n < F_k$.\\
Base: $n = 0 \rightarrow 0, n = 1 \rightarrow 1, n = 2 \rightarrow 10$.\\
Induction step: If $F_k \le n < F_{k + 1}$ and because
$F_{k + 1} = F_k + F_{k - 1}$, we have $n - F_k < F_{k - 1}$. Because we
can write $F_{k - 1}$ correct by our hypothesis as
$F_{k - 1} = \sum_{i = 0}^{k - 2} n_i F_i$, we can write
$n = F_k + \sum_{i = 0}^{k - 2} n_i F_i$.\\
\\
We still need to prove that we can do this in a unique way:\\
Let $n \in \mathbb{N}$ be the smallest number for which there is more than one
representation. We must assume $n_k \ne n'_k$, otherwise $n - F_k$ would also
have more than one representation. The maximum number we can represent, of
length $k - 1$ is 101010..., which is smaller than $F_k$
($n^{\mathrm{max}} = F_{k - 1} + F_{k - 3} + ... < F_k$). \qed\\
\\
Since this system can be used to enumerate $n \in \mathbb{N}$ and it has the
property that no two ones succeed each other, this is the mapping we are
looking for.\\
\\
If we now write the partitions of $\mathbb{F}$:
\begin{eqnarray*}
  \mathbb{F}_a &=& \{0, 2, 3, 5, 7, 8, 10, 11, 13, 15, ...\}\\
  \mathbb{F}_b &=& \{1, 4, 6, 9, 12, 14, 17, 19, 22, 25, ...\}
\end{eqnarray*}

\paragraph{Proof} We have:
\begin{eqnarray*}
  \mathbb{F}_a &=& \{n \in \mathbb{N}, \mathrm{Fib}(n) \in \{0, 1\}^* 0\}\\
  \mathbb{F}_b &=& \{n \in \mathbb{N}, \mathrm{Fib}(n) \in \{0, 1\}^* 1\}
\end{eqnarray*}
In other words, we get an $a$ at position $n$ if $n$ ends with a 0 in the
Fibonacci expansion, we get an $b$ otherwise. Hence the similarity with
$(ab)^{\mathbb{N}}$. \qed\\
\\
Let us look at the first few substitutions of $\sigma$:
\begin{verbatim}
  u(0) = a
  u(1) = ab
  u(2) = aba
  u(3) = abaab
  u(4) = abaababa
\end{verbatim}

\paragraph{Theorem} $u(n) = u(n - 1) u(n - 2), n \in \mathbb{N}, n \ge 2$

\paragraph{Proof} By induction:\\
Base: $u(2) = aba = u(1) u(0), u(3) = abaab = u(2) u(1)$.\\
Induction step: $u(n + 1) = u(n) u(n - 1)$.\\
We know that $u(n + 1) = \sigma(u(n))$, so we can re-write this as:
$u(n + 1) = \sigma(u(n - 1) u(n - 2)$, because of our hypothesis. This
equals $\sigma(u(n - 1)) \sigma(u(n - 2)) = u(n) u(n - 1)$. \qed
\begin{verbatim}
  a|b|a|ab|aba|abaab ...
  0|1|1|11|111|11111
      0|00|000|00000
        01|001|00011
           010|00100
               01001
\end{verbatim}
We can see that the Fibonacci expansions of the integers in $u(n)$ are all of
length $n$ and we can deduce from the definition of the Fibonacci expansion
that each string $...0 \rightarrow ...00, ...01$, and $...1 \rightarrow ...10$
in a 1-1 correspondence with the fixed point.\\

\section*{Unknown territory}
Just for the fun of it, look at the following substitution:
$\sigma : \sigma(a) \rightarrow aba, \sigma(b) \rightarrow bb$, this looks
like the Cantor substitution, but like the Fibonacci substitution it is not of
fixed length.\\
\\
This is the associated automaton:\\
\begin{graph}(0, 3)(-4, -1.5)
  \graphnodecolour{1}
  \graphnodesize{1}
  \roundnode{s1}(-2, 0) \nodetext{s1}(0, 0){$a$}
  \roundnode{s2}(0, 0)  \nodetext{s2}(0, 0){$b$}

  \dirloopedge{s1}{50}(-1, 0) \freetext(-3.6, 0){0,2}
  \diredge{s1}{s2} \edgetext{s1}{s2}{1}
  \dirloopedge{s2}{50}(1, 0) \freetext(1.6, 0){0,1}

  \freetext(-2, -0.7){$\Uparrow$}
\end{graph}\\
Now we have to find an enumeration system in base 3 that has the following
property: if $n_i = 1$ then $\not\exists (j > i) \{ n_j = 2\}$. This follows
from the fact that this automaton recognises the language:
$\{0, 2\}^* \{\epsilon, (1\{0, 1\}^*)\}$.\\
Its first few elements are:
$\{0, 1, 2, 10, 11, 20, 21, 22, 100, 101, 110, 111, 200\}$ and we left out:
$\{12, 102, 112, 120, 121, 122\}$. Perhaps there is a function that maps
this sequence to $\mathbb{N}$, if so (as in the Fibonacci case), we can feed
our automaton with it.\\
\\
In general we can always find the language associated with the automaton. And
we can always enumerate the words in this language (by length and
lexicographically). However, since in general we cannot give the $n$--th word
in the language a priory, it is all pretty useless.
\end{document}
