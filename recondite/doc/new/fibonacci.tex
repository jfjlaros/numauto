\documentclass{article}
\usepackage{/home/jeroen/studie/dw/graphs/graphs}
\usepackage{amsfonts, amssymb}
\frenchspacing
\begin{document}

\newcommand{\qed}{$\blacksquare$}

\section*{}

\section*{The Fibonacci substitution}
The Fibonacci substitution:
\begin{displaymath}
\sigma: \left\{ \begin{array}{l}
a \rightarrow ab\\
b \rightarrow a
\end{array} \right.
\end{displaymath}
With $a$ as the initial letter gives the following fixed point:
\begin{verbatim}
  u = abaababaabaababaababaabaababaabaababaababaabaababaaba...
\end{verbatim}
This is the automaton associated with the substitution:\\
\begin{graph}(0, 3)(-4, -1.5)
  \graphnodecolour{1}
  \graphnodesize{1}
  \roundnode{s1}(-2, 0) \nodetext{s1}(0, 0){$a$}
  \roundnode{s2}(0, 0)  \nodetext{s2}(0, 0){$b$}

  \dirloopedge{s1}{50}(-1, 0) \freetext(-3.6, 0){0}
  \dirbow{s1}{s2}{.2} \bowtext{s1}{s2}{.2}{1}
  \dirbow{s2}{s1}{.2} \bowtext{s2}{s1}{.2}{0}

  \freetext(-2, -0.7){$\Uparrow$}
\end{graph}\\
To make this automaton work, we need to feed it with the Zeckendorf expansion
of an integer $n$.\\
\paragraph{The Fibonacci sequence}
Let $(F_n)_{n \in \mathbb{N}}$ be the sequence of integers defined by $F_0 = 1,
F_1 = 2$ and for any integer $n > 1, F_{n + 1} = F_{n - 1} + F_n$.\\
\paragraph{The Zeckendorf expansion}
If $n = \sum_{i = 0}^k n_i F_i$ with $n_k = 1, n_i \in \{0, 1\}$ and
$\forall (i < k) \{n_i n_{i + 1} = 0\}$, we say that
Zeck$(n) = n_k n_{k - 1} ... n_0 \in \{0, 1\}^{k + 1}$ is the Zeckendorf
expansion of the integer $n$.\\
\\
If we write the partitions of
$\mathbb{F}$ as $\mathbb{F}_a$ and $\mathbb{F}_b$,
\begin{eqnarray*}
  \mathbb{F}_a &=& \{0, 2, 3, 5, 7, 8, 10, 11, 13, 15, ...\}\\
  \mathbb{F}_b &=& \{1, 4, 6, 9, 12, 14, 17, 19, 22, 25, ...\}
\end{eqnarray*}
Then
\begin{eqnarray*}
  \mathbb{F}_a &=& \{n \in \mathbb{N}, \mathrm{Zeck}(n) \in \{0, 1\}^* 0\}\\
  \mathbb{F}_b &=& \{n \in \mathbb{N}, \mathrm{Zeck}(n) \in \{0, 1\}^* 1\}
\end{eqnarray*}
Hence we get an $a$ at position $n$ if $n$ ends with a 0 in the
Zeckendorf expansion, we get an $b$ otherwise.

\paragraph{Fibonacci's `brother'}
There seems to be a class of substitutions for which we can find an enumeration
system. Here is a little example:
\begin{displaymath}
\sigma: \left\{ \begin{array}{l}
a \rightarrow ab\\
b \rightarrow b
\end{array} \right.
\end{displaymath}
\begin{graph}(0, 3)(-4, -1.5)
  \graphnodecolour{1}
  \graphnodesize{1}
  \roundnode{s1}(-2, 0) \nodetext{s1}(0, 0){$a$}
  \roundnode{s2}(0, 0)  \nodetext{s2}(0, 0){$b$}

  \dirloopedge{s1}{50}(-1, 0) \freetext(-3.6, 0){0}
  \diredge{s1}{s2} \edgetext{s1}{s2}{1}
  \dirloopedge{s2}{50}(1, 0) \freetext(1.6, 0){0}

  \freetext(-2, -0.7){$\Uparrow$}
\end{graph}\\
The enumeration system for this substitution is based on the sequence of 
integers in $\mathbb{N}^+$. So the first few expansions are as follows:
\begin{verbatim}
0 -> 0
1 -> 1
2 -> 10
3 -> 100
4 -> 1000
 ...
\end{verbatim}

\section*{A generalization}
But can we find a more general way for this phenomenon? The answer seems to
be yes for a (possibly small) class of substitutions. This class can be found
by calculating the sequence on which the expansion is based from the 
substitution itself. If we look at the Fibonacci substitution, we see that for
each word $\sigma_n$:
\begin{eqnarray*}
|\sigma_n|_a &=& |\sigma_{n - 1}|_a + |\sigma_{n - 1}|_b\\
|\sigma_n|_b &=& |\sigma_{n - 1}|_a
\end{eqnarray*}
Or in matrix form:
\begin{displaymath} \left( \begin{array}{cc}
1 & 1\\
1 & 0
\end{array} \right) \end{displaymath}
If we now define the initial matrix as:
\begin{displaymath} \left( \begin{array}{c}
1 \\
0 
\end{array} \right) \end{displaymath}
and multiply it repeatedly with the (2 $\times$ 2) matrix, we obtain:
\begin{displaymath} 
\left( \begin{array}{c}
1 \\
0 
\end{array} \right),
\left( \begin{array}{c}
1 \\
1 
\end{array} \right),
\left( \begin{array}{c}
2 \\
1 
\end{array} \right),
\left( \begin{array}{c}
3 \\
2 
\end{array} \right),
\left( \begin{array}{c}
5 \\
3 
\end{array} \right),
\left( \begin{array}{c}
8 \\
5 
\end{array} \right),
\left( \begin{array}{c}
13 \\
8 
\end{array} \right)
\end{displaymath}
And if we now add the rows of the matrices we obtain the Fibonacci sequence.
The matrix for the `brother' of the Fibonacci sequence is:
\begin{displaymath} \left( \begin{array}{cc}
1 & 0\\
1 & 1
\end{array} \right) \end{displaymath}
Which applied to the initial matrix yields the sequence of integers in $\mathbb{N}^+$.\\
By using this method we can cover all $k$-automata, the Fibonacci automaton and
its `brother' and, as we shall see, at least one one other class of automata.

%\paragraph{}
%The Tribonacci substitution:
%\begin{displaymath}
%\sigma: \left\{ \begin{array}{l}
%a \rightarrow ab\\
%b \rightarrow ac\\
%c \rightarrow b
%\end{array} \right.
%\end{displaymath}
%With $a$ as the initial letter gives the following fixed point:
%\begin{verbatim}
%  u = abacabbabacacabacabbabbabacabbabacacabacacabacabbabac...
%\end{verbatim}
%This is the automaton associated with the substitution:\\
%\begin{graph}(0, 3)(-4, -1.5)
%  \graphnodecolour{1}
%  \graphnodesize{1}
%  \roundnode{s1}(-2, 0) \nodetext{s1}(0, 0){$a$}
%  \roundnode{s2}(0, 0)  \nodetext{s2}(0, 0){$b$}
%  \roundnode{s3}(2, 0)  \nodetext{s3}(0, 0){$c$}
%
%  \dirloopedge{s1}{50}(-1, 0) \freetext(-3.6, 0){0}
%  \dirbow{s1}{s2}{.2} \bowtext{s1}{s2}{.2}{1}
%  \dirbow{s2}{s1}{.2} \bowtext{s2}{s1}{.2}{0}
%  \dirbow{s2}{s3}{.2} \bowtext{s2}{s3}{.2}{1}
%  \dirbow{s3}{s2}{.2} \bowtext{s3}{s2}{.2}{0}
%
%  \freetext(-2, -0.7){$\Uparrow$}
%\end{graph}
%\paragraph{The Tribonacci sequence}
%Let $(F_n)_{n \in \mathbb{N}}$ be the sequence of integers defined by $F_0 = 1,
%F_1 = 2, F_2 = 4$ and for any integer
%$n > 2, F_{n + 1} = F_{n - 2} + F_{n - 1} + F_n$.\\
%\paragraph{The Tribonacci-Zeckendorf expansion}
%If $n = \sum_{i = 0}^k n_i F_i$ with $n_k = 1, n_i \in \{0, 1\}$ and
%$\forall (i < k) \{n_i n_{i + 1} n_{i + 2} = 0\}$, we say that
%T-Zeck$(n) = n_k n_{k - 1} ... n_0 \in \{0, 1\}^{k + 1}$ is the 
%Tribonacci-Zeckendorf expansion of the integer $n$.\\
%\\
%We can now feed our automaton with this expansion to get the fixed point again.
%
%\section*{The 4-bonacci substitution}
%The 4-bonacci substitution:
%\begin{displaymath}
%\sigma: \left\{ \begin{array}{l}
%a \rightarrow ab\\
%b \rightarrow ac\\
%c \rightarrow bd\\
%d \rightarrow c
%\end{array} \right.
%\end{displaymath}
%With $a$ as the initial letter gives the following fixed point:
%\begin{verbatim}
%  u = abacabbdabacaccabacabbdabbdbdabacabbdabacaccabacaccac...
%\end{verbatim}
%This is the automaton associated with the substitution:\\
%\begin{graph}(0, 3)(-4, -1.5)
%  \graphnodecolour{1}
%  \graphnodesize{1}
%  \roundnode{s1}(-2, 0) \nodetext{s1}(0, 0){$a$}
%  \roundnode{s2}(0, 0)  \nodetext{s2}(0, 0){$b$}
%  \roundnode{s3}(2, 0)  \nodetext{s3}(0, 0){$c$}
%  \roundnode{s4}(4, 0)  \nodetext{s4}(0, 0){$d$}
%
%  \dirloopedge{s1}{50}(-1, 0) \freetext(-3.6, 0){0}
%  \dirbow{s1}{s2}{.2} \bowtext{s1}{s2}{.2}{1}
%  \dirbow{s2}{s1}{.2} \bowtext{s2}{s1}{.2}{0}
%  \dirbow{s2}{s3}{.2} \bowtext{s2}{s3}{.2}{1}
%  \dirbow{s3}{s2}{.2} \bowtext{s3}{s2}{.2}{0}
%  \dirbow{s3}{s4}{.2} \bowtext{s3}{s4}{.2}{1}
%  \dirbow{s4}{s3}{.2} \bowtext{s4}{s3}{.2}{0}
%
%  \freetext(-2, -0.7){$\Uparrow$}
%\end{graph}
%\paragraph{The 4-bonacci sequence}
%Let $(F_n)_{n \in \mathbb{N}}$ be the sequence of integers defined by $F_0 = 1,
%F_1 = 2, F_2 = 4, F_3 = 8$ and for any integer
%$n > 3, F_{n + 1} = F_{n - 3} + F_{n - 2} + F_{n - 1} + F_n$.\\
%\paragraph{The 4-bonacci-Zeckendorf expansion}
%If $n = \sum_{i = 0}^k n_i F_i$ with $n_k = 1, n_i \in \{0, 1\}$ and
%$\forall (i < k) \{n_i n_{i + 1} n_{i + 2} n_{i + 3} = 0\}$, we say that
%4-Zeck$(n) = n_k n_{k - 1} ... n_0 \in \{0, 1\}^{k + 1}$ is the 
%4-bonacci-Zeckendorf expansion of the integer $n$.
%
%\section*{Other substitutions that have a simmilar property}
%All previous examples also seem to work for other substitutions of length two
%as long as the last substitution is of length 1, so a substitution that looks
%like this:
%\begin{displaymath}
%\sigma: \left\{ \begin{array}{l}
%a \rightarrow ab\\
%b \rightarrow ac\\
%c \rightarrow a
%\end{array} \right.
%\end{displaymath}
%will have an associated automaton that can be fed with the 
%Tribonacci-Zeckendorf expansion of an integer.\\
%\\
%Now consider:
%\begin{displaymath}
%\sigma: \left\{ \begin{array}{l}
%a \rightarrow ab\\
%b \rightarrow ac\\
%c \rightarrow a
%\end{array} \right.
%\end{displaymath}
%
%\section*{Generalization}
%The claim is that this method works for all substitutions of length 2 where
%the last substitution is of length 1.\\
%\\
%Maybe we can also find a method that works for all substitutions of length
%2 where one substitution is of length 1.
%
\end{document}
